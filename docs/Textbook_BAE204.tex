\documentclass[]{book}
\usepackage{lmodern}
\usepackage{amssymb,amsmath}
\usepackage{ifxetex,ifluatex}
\usepackage{fixltx2e} % provides \textsubscript
\ifnum 0\ifxetex 1\fi\ifluatex 1\fi=0 % if pdftex
  \usepackage[T1]{fontenc}
  \usepackage[utf8]{inputenc}
\else % if luatex or xelatex
  \ifxetex
    \usepackage{mathspec}
  \else
    \usepackage{fontspec}
  \fi
  \defaultfontfeatures{Ligatures=TeX,Scale=MatchLowercase}
\fi
% use upquote if available, for straight quotes in verbatim environments
\IfFileExists{upquote.sty}{\usepackage{upquote}}{}
% use microtype if available
\IfFileExists{microtype.sty}{%
\usepackage{microtype}
\UseMicrotypeSet[protrusion]{basicmath} % disable protrusion for tt fonts
}{}
\usepackage[margin=1in]{geometry}
\usepackage{hyperref}
\hypersetup{unicode=true,
            pdftitle={BAE 204: Introduction to Environmental and Ecological Engineering},
            pdfauthor={François Birgand},
            pdfborder={0 0 0},
            breaklinks=true}
\urlstyle{same}  % don't use monospace font for urls
\usepackage{natbib}
\bibliographystyle{apalike}
\usepackage{color}
\usepackage{fancyvrb}
\newcommand{\VerbBar}{|}
\newcommand{\VERB}{\Verb[commandchars=\\\{\}]}
\DefineVerbatimEnvironment{Highlighting}{Verbatim}{commandchars=\\\{\}}
% Add ',fontsize=\small' for more characters per line
\usepackage{framed}
\definecolor{shadecolor}{RGB}{248,248,248}
\newenvironment{Shaded}{\begin{snugshade}}{\end{snugshade}}
\newcommand{\KeywordTok}[1]{\textcolor[rgb]{0.13,0.29,0.53}{\textbf{#1}}}
\newcommand{\DataTypeTok}[1]{\textcolor[rgb]{0.13,0.29,0.53}{#1}}
\newcommand{\DecValTok}[1]{\textcolor[rgb]{0.00,0.00,0.81}{#1}}
\newcommand{\BaseNTok}[1]{\textcolor[rgb]{0.00,0.00,0.81}{#1}}
\newcommand{\FloatTok}[1]{\textcolor[rgb]{0.00,0.00,0.81}{#1}}
\newcommand{\ConstantTok}[1]{\textcolor[rgb]{0.00,0.00,0.00}{#1}}
\newcommand{\CharTok}[1]{\textcolor[rgb]{0.31,0.60,0.02}{#1}}
\newcommand{\SpecialCharTok}[1]{\textcolor[rgb]{0.00,0.00,0.00}{#1}}
\newcommand{\StringTok}[1]{\textcolor[rgb]{0.31,0.60,0.02}{#1}}
\newcommand{\VerbatimStringTok}[1]{\textcolor[rgb]{0.31,0.60,0.02}{#1}}
\newcommand{\SpecialStringTok}[1]{\textcolor[rgb]{0.31,0.60,0.02}{#1}}
\newcommand{\ImportTok}[1]{#1}
\newcommand{\CommentTok}[1]{\textcolor[rgb]{0.56,0.35,0.01}{\textit{#1}}}
\newcommand{\DocumentationTok}[1]{\textcolor[rgb]{0.56,0.35,0.01}{\textbf{\textit{#1}}}}
\newcommand{\AnnotationTok}[1]{\textcolor[rgb]{0.56,0.35,0.01}{\textbf{\textit{#1}}}}
\newcommand{\CommentVarTok}[1]{\textcolor[rgb]{0.56,0.35,0.01}{\textbf{\textit{#1}}}}
\newcommand{\OtherTok}[1]{\textcolor[rgb]{0.56,0.35,0.01}{#1}}
\newcommand{\FunctionTok}[1]{\textcolor[rgb]{0.00,0.00,0.00}{#1}}
\newcommand{\VariableTok}[1]{\textcolor[rgb]{0.00,0.00,0.00}{#1}}
\newcommand{\ControlFlowTok}[1]{\textcolor[rgb]{0.13,0.29,0.53}{\textbf{#1}}}
\newcommand{\OperatorTok}[1]{\textcolor[rgb]{0.81,0.36,0.00}{\textbf{#1}}}
\newcommand{\BuiltInTok}[1]{#1}
\newcommand{\ExtensionTok}[1]{#1}
\newcommand{\PreprocessorTok}[1]{\textcolor[rgb]{0.56,0.35,0.01}{\textit{#1}}}
\newcommand{\AttributeTok}[1]{\textcolor[rgb]{0.77,0.63,0.00}{#1}}
\newcommand{\RegionMarkerTok}[1]{#1}
\newcommand{\InformationTok}[1]{\textcolor[rgb]{0.56,0.35,0.01}{\textbf{\textit{#1}}}}
\newcommand{\WarningTok}[1]{\textcolor[rgb]{0.56,0.35,0.01}{\textbf{\textit{#1}}}}
\newcommand{\AlertTok}[1]{\textcolor[rgb]{0.94,0.16,0.16}{#1}}
\newcommand{\ErrorTok}[1]{\textcolor[rgb]{0.64,0.00,0.00}{\textbf{#1}}}
\newcommand{\NormalTok}[1]{#1}
\usepackage{longtable,booktabs}
\usepackage{graphicx,grffile}
\makeatletter
\def\maxwidth{\ifdim\Gin@nat@width>\linewidth\linewidth\else\Gin@nat@width\fi}
\def\maxheight{\ifdim\Gin@nat@height>\textheight\textheight\else\Gin@nat@height\fi}
\makeatother
% Scale images if necessary, so that they will not overflow the page
% margins by default, and it is still possible to overwrite the defaults
% using explicit options in \includegraphics[width, height, ...]{}
\setkeys{Gin}{width=\maxwidth,height=\maxheight,keepaspectratio}
\IfFileExists{parskip.sty}{%
\usepackage{parskip}
}{% else
\setlength{\parindent}{0pt}
\setlength{\parskip}{6pt plus 2pt minus 1pt}
}
\setlength{\emergencystretch}{3em}  % prevent overfull lines
\providecommand{\tightlist}{%
  \setlength{\itemsep}{0pt}\setlength{\parskip}{0pt}}
\setcounter{secnumdepth}{5}
% Redefines (sub)paragraphs to behave more like sections
\ifx\paragraph\undefined\else
\let\oldparagraph\paragraph
\renewcommand{\paragraph}[1]{\oldparagraph{#1}\mbox{}}
\fi
\ifx\subparagraph\undefined\else
\let\oldsubparagraph\subparagraph
\renewcommand{\subparagraph}[1]{\oldsubparagraph{#1}\mbox{}}
\fi

%%% Use protect on footnotes to avoid problems with footnotes in titles
\let\rmarkdownfootnote\footnote%
\def\footnote{\protect\rmarkdownfootnote}

%%% Change title format to be more compact
\usepackage{titling}

% Create subtitle command for use in maketitle
\newcommand{\subtitle}[1]{
  \posttitle{
    \begin{center}\large#1\end{center}
    }
}

\setlength{\droptitle}{-2em}
  \title{BAE 204: Introduction to Environmental and Ecological Engineering}
  \pretitle{\vspace{\droptitle}\centering\huge}
  \posttitle{\par}
  \author{François Birgand}
  \preauthor{\centering\large\emph}
  \postauthor{\par}
  \predate{\centering\large\emph}
  \postdate{\par}
  \date{2018-02-20}

\usepackage{booktabs}
\usepackage{amsthm}
\makeatletter
\def\thm@space@setup{%
  \thm@preskip=8pt plus 2pt minus 4pt
  \thm@postskip=\thm@preskip
}
\makeatother

\usepackage{amsthm}
\newtheorem{theorem}{Theorem}[chapter]
\newtheorem{lemma}{Lemma}[chapter]
\theoremstyle{definition}
\newtheorem{definition}{Definition}[chapter]
\newtheorem{corollary}{Corollary}[chapter]
\newtheorem{proposition}{Proposition}[chapter]
\theoremstyle{definition}
\newtheorem{example}{Example}[chapter]
\theoremstyle{definition}
\newtheorem{exercise}{Exercise}[chapter]
\theoremstyle{remark}
\newtheorem*{remark}{Remark}
\newtheorem*{solution}{Solution}
\begin{document}
\maketitle

{
\setcounter{tocdepth}{1}
\tableofcontents
}
\chapter*{Preface}\label{preface}
\addcontentsline{toc}{chapter}{Preface}

This is the first attempt at a textbook for BAE 204. It will be updated
as class proceeds.

\section*{Author}\label{author}
\addcontentsline{toc}{section}{Author}

\href{https://francoisbirgand.github.io}{François Birgand} is an
Associate Professor of biogeochemistry and ecological engineering at
North Carolina State University, in Raleigh, NC, USA. His research
interests aim to provide solutions to improve the quality of water in
streams and rivers. Practically, his research focuses on improving the
treatment efficiencies of streams, wetlands, soils or woodchip
bioreactors. Much of this research is based upon principles this
textbook is trying to provide for students of all ages.

\chapter{Introduction}\label{intro}

You can label chapter and section titles using \texttt{\{\#label\}}
after them, e.g., we can reference Chapter \ref{intro}. If you do not
manually label them, there will be automatic labels anyway, e.g.,
Chapter \ref{methods}.

Figures and tables with captions will be placed in \texttt{figure} and
\texttt{table} environments, respectively.

\begin{Shaded}
\begin{Highlighting}[]
\KeywordTok{par}\NormalTok{(}\DataTypeTok{mar =} \KeywordTok{c}\NormalTok{(}\DecValTok{4}\NormalTok{, }\DecValTok{4}\NormalTok{, .}\DecValTok{1}\NormalTok{, .}\DecValTok{1}\NormalTok{))}
\KeywordTok{plot}\NormalTok{(pressure, }\DataTypeTok{type =} \StringTok{'b'}\NormalTok{, }\DataTypeTok{pch =} \DecValTok{19}\NormalTok{)}
\end{Highlighting}
\end{Shaded}

\begin{figure}

{\centering \includegraphics[width=0.8\linewidth]{Textbook_BAE204_files/figure-latex/nice-fig-1} 

}

\caption{Here is a nice figure!}\label{fig:nice-fig}
\end{figure}

Reference a figure by its code chunk label with the \texttt{fig:}
prefix, e.g., see Figure \ref{fig:nice-fig}. Similarly, you can
reference tables generated from \texttt{knitr::kable()}, e.g., see Table
\ref{tab:nice-tab}.

\begin{Shaded}
\begin{Highlighting}[]
\NormalTok{knitr}\OperatorTok{::}\KeywordTok{kable}\NormalTok{(}
  \KeywordTok{head}\NormalTok{(iris, }\DecValTok{20}\NormalTok{), }\DataTypeTok{caption =} \StringTok{'Here is a nice table!'}\NormalTok{,}
  \DataTypeTok{booktabs =} \OtherTok{TRUE}
\NormalTok{)}
\end{Highlighting}
\end{Shaded}

\begin{table}

\caption{\label{tab:nice-tab}Here is a nice table!}
\centering
\begin{tabular}[t]{rrrrl}
\toprule
Sepal.Length & Sepal.Width & Petal.Length & Petal.Width & Species\\
\midrule
5.1 & 3.5 & 1.4 & 0.2 & setosa\\
4.9 & 3.0 & 1.4 & 0.2 & setosa\\
4.7 & 3.2 & 1.3 & 0.2 & setosa\\
4.6 & 3.1 & 1.5 & 0.2 & setosa\\
5.0 & 3.6 & 1.4 & 0.2 & setosa\\
\addlinespace
5.4 & 3.9 & 1.7 & 0.4 & setosa\\
4.6 & 3.4 & 1.4 & 0.3 & setosa\\
5.0 & 3.4 & 1.5 & 0.2 & setosa\\
4.4 & 2.9 & 1.4 & 0.2 & setosa\\
4.9 & 3.1 & 1.5 & 0.1 & setosa\\
\addlinespace
5.4 & 3.7 & 1.5 & 0.2 & setosa\\
4.8 & 3.4 & 1.6 & 0.2 & setosa\\
4.8 & 3.0 & 1.4 & 0.1 & setosa\\
4.3 & 3.0 & 1.1 & 0.1 & setosa\\
5.8 & 4.0 & 1.2 & 0.2 & setosa\\
\addlinespace
5.7 & 4.4 & 1.5 & 0.4 & setosa\\
5.4 & 3.9 & 1.3 & 0.4 & setosa\\
5.1 & 3.5 & 1.4 & 0.3 & setosa\\
5.7 & 3.8 & 1.7 & 0.3 & setosa\\
5.1 & 3.8 & 1.5 & 0.3 & setosa\\
\bottomrule
\end{tabular}
\end{table}

You can write citations, too. For example, we are using the
\textbf{bookdown} package \citep{R-bookdown} in this sample book, which
was built on top of R Markdown and \textbf{knitr} \citep{xie2015}.

\chapter{Life's secrets for capturing
energy}\label{lifes-secrets-for-capturing-energy}

Much of the environmental and ecological engineering challenges are
about dealing with too many suspended sediments and too many nutrients
in surface and ground waters, and the consequences of too much organic
matter and reduced compounds in aquatic ecosystems. We have also seen
that these three challenges are inter-connected and related. Nutrients
and organic matter are related because the former contain the needed
atoms necessary to build living cell and organism structures,
corresponding to \emph{living organic matter}, while the latter
generally refers to \emph{dead organic matter} and/or the assemblage of
molecules which at one point were part of a living organism.

This definition of the linkage between the two is hardly satisfying,
however\ldots{} When one looks at where the electrons are allocated on
the important atoms, a much more unifying scheme appears. This chapter
describes this unifying scheme.

\section{The five fundamental requirements of
life}\label{the-five-fundamental-requirements-of-life}

Before we dive down to the electron level, let us make sure we recognize
that it is possible to simplify why there is, or not, life on our planet
and the universe. Such a list includes:

\begin{itemize}
\tightlist
\item
  the presence of liquid water
\item
  available nutrients that can fulfill the need for the six most
  numerous and important atoms that build most of our cell and organism
  structure: C, H, O, N, S, P. Without the commas comes this delightful
  acronym \emph{CHONSP} which generations of students have come to love
\item
  a source of energy, which in most cases is the form of solar or
  chemical energy
\item
  electrons acceptor(s), without which chemical energy cannot be
  released
\item
  and a suitable temperature range (about -5°C to +50°C) because of
  course most of living organisms will not live for very long outside
  this range
\end{itemize}

There are many other requirements for life to occur, e.g., the ability
to reproduce or even to die, but for what we are interested, this is a
satisfying list. In this chapter, we will address the source of energy
and what secret living organisms have found to store energy. Primary
producers, which include most of chlorophyll containing plants from
algae to
\href{https://en.wikipedia.org/wiki/Flowering_plant}{angiosperms}
(flowering plants), have the ability to use solar radiations as a source
of energy. An entire chapter is dedicated to this marvel.

However, even for primary producers, solar radiations cannot be the sole
source of energy, otherwise they would die at night\ldots{} And for the
rest of living organisms, solar radiations are just not a source of
energy (merely a source of `bien-être' or vitamin D for
humans\ldots{}!). So obviously, life has had to find a solution so that
energy would be available for all conditions of light and temperature on
earth.

The \emph{first secret of life} is the ability for \textbf{store energy
in the form of high energy electrons onto organic molecules}. Having a
source of energy is always a good thing. However, energy can only be
released if there is an outlet for it, otherwise it stays as potential
energy. Practically, this means that the energy stored on organic
molecules can and is only released when the electrons go from the high
potential to a lower one, or in others words from \emph{electron donors}
(organic molecules) to \emph{electron acceptors}.

This is \emph{the second secret} of life on earth: on our planet exists
this miracle molecule, O\textsubscript{2}, which acts as an extremely
powerful electron acceptor. So you have to imagine O\textsubscript{2}
less as a \emph{gentle acceptor of electrons} that organic matter would
be kind enough to \emph{donate}, but more like a very \emph{aggressive
electron seeker} and any organic matter located close to oxygen runs the
risk to be \textbf{\protect\hyperlink{oxidation}{oxidized}}, that is to
lose its electrons. Sometimes, I like to refer to O\textsubscript{2} as
the \emph{electron kleptomaniac}. So in oxygenated environments such as
earth's atmosphere and most water bodies, living organisms' only concern
is to have potential energy available in the form of high energy
electrons stored onto organic molecules, because this energy can easily
be released at any time thanks to the very oxidizing agent
O\textsubscript{2}. Chapter \ref{aerobic-and-anaerobic-respiration} goes
into the details of how this energy is released and transferred in
cells.

\section{Electron allocation onto
CHONSP}\label{electron-allocation-onto-chonsp}

First, among the 6 atoms that form CHONSP, and this is true for all
atoms except for \href{https://en.wikipedia.org/wiki/Noble_gas}{noble
gases} which are stable
\href{https://en.wikipedia.org/wiki/Monatomic_gas}{monatomic atoms},
none of them exist as monatomic atoms: they always form bonds with other
atoms to form molecules. Among the CHONSP, three of them are
\href{https://en.wikipedia.org/wiki/Diatomic_molecule}{homonuclear
diatomic molecules}, i.e., they can form molecules of two atoms of the
same chemical element: H\textsubscript{2} (dihydrogen, although it is
most often referred to simply as hydrogen, which can be quite
confusing), O\textsubscript{2} (dioxygen, although oxygen is also most
often used, unfortunately), and N\textsubscript{2} (dinitrogen, which
name is generally properly applied). Obviously, they can also bound to
other elements, which is what we are about to see.

If indeed, on our planet floats a very oxidizing agent in the form of
O\textsubscript{2}, then the stable state of all other molecules should
be where most other elements have lost their electrons to oxygen. And
there ought to be techniques to see where the electrons are allocated on
molecules.

Chemists have created the \textbf{oxidation state} (\emph{OS}) or
\textbf{oxidation number} indicator which quantifies this electron
allocation to some extent, but we find it to be rather confusing. For
example if C, N, and S would have lost \emph{all} their electrons, their
OS would differ\ldots{} In the first case, OS\textsubscript{C} = -4,
OS\textsubscript{N} = -5, and OS\textsubscript{S} = -6, because the OS
indicator is relative to the number of valence electrons in the free
form of the atoms. More \protect\hyperlink{oxidation-state}{discussion
and information on Oxidation State} is available in the
\protect\hyperlink{glossary}{glossary} of this book.

\emph{This chapter is still under construction}

\chapter{Aerobic and anaerobic
respiration}\label{aerobic-and-anaerobic-respiration}

\section{Generating energy: transfer of
electrons}\label{generating-energy-transfer-of-electrons}

Us humans have the freedom to go about as we wish. All we need is to
eat, drink, breath, and eliminate what we do not need. We could take our
automobile cars as another example of bodies able to move about as long
as they are fed some fuel. A lot of energy is realized from combustion
of fuel or wood. In fact, provided that this energy be funneled into
some very smart mechanical systems (e.g., combustion engines), and the
energy power delivered is largely superior to any power any human could
output. However, the energy consumed is also phenomenal and much of it
is lost through heat. Life has found a way to be a lot more parsimonious
with energy spending, which makes living organisms to be a lot more
energy efficient than combustion in the long run: the \emph{respiration}
process.

But in both the cases of combustion or of respiration, energy is
liberated or transferred to \protect\hyperlink{ATP}{ATP}, respectively,
when \emph{\textbf{electrons are transferred}} from an
\emph{\textbf{electron donor}} to an \emph{\textbf{electron acceptor}}.
Not surprisingly, life and its diversity as we know it, has been able to
develop on our planet because primary producers have been able to
transfer and store solar energy into a chemical energy, where electrons
are effectively stored onto organic molecules, and, because dioxygen
O\textsubscript{2}, the most powerful electron acceptor, is freely
floating in our atmosphere.

But, then if we ourselves are nothing but a lot of organic matter,
therefore a very large source of electrons, and if we leave in an
oxygenated environment, how come we are not going into combustion?
Obviously this applies to all living organisms. Some of you might think
that it is because of the water. But then why would not a very dry piece
of paper, or leaves, not automatically catch on fire? The answer lays in
two words:
\href{https://en.wikipedia.org/wiki/Activation_energy}{activation
energy}.

Electrons do not readily transfer from an electron donor to an electron
acceptor because of the activation energy barrier which must be
overcome. Activation energy is the level of energy necessary to be
overcome for a reaction to carry on. On the figure below borrowed from
Wikipedia, the energy difference between the peaks and the final energy
level corresponds to the activation energy E\textsubscript{a}

\begin{figure}

{\centering \includegraphics[width=0.7\linewidth]{pictures/Activation_energy} 

}

\caption{Activation Energy *(E~a~)* needed for a reaction to proceed. *(E~a~)* can be lowered thanks to a catalyst, and in the case of the respiration chain of living organisms, many of them. Figure borrowed from Wikipedia, Copyrighted free use, https://commons.wikimedia.org/w/index.php?curid=779552}\label{fig:EnergyActivation}
\end{figure}

In normal conditions, this energy is preventing a reaction to proceed.
This is why wood or most organic matter do not go into combustion or
decomposition on their own. Nitroglycerin is notoriously unstable and
the activation energy for electrons to be transferred to
O\textsubscript{2} is very small, hence its propensity to explode and
its danger. However, the activation energy can be overcome with enough
heat brought near an electron donor in the form of a flame or a spark.
Not much heat is necessary for natural gas (CH\textsubscript{4}) for
combustion to take place, i.e., for transfer of electrons from the
carbon to oxygen, releasing twice the volume of gases:

\begin{equation}
CH_4 + O_2 \Rightarrow CO_2 + H_2O
\label{eq:CH4combustion}
\end{equation}

hence the explosive nature of this reaction.

\section{Generating energy without
combustion}\label{generating-energy-without-combustion}

Now because none of us and all the living organisms with us, \emph{burn}
to generate energy, there must be other systems to manage to liberate
energy, and, there must be ways to have much lower activation energy.
Life has found several mechanisms to optimize the transfer of energy
from organic matter to the a magic molecule:
\protect\hyperlink{ATP}{ATP} or Adenosine TriPhosphate.

\section{ATP or the energy currency of the
cell}\label{atp-or-the-energy-currency-of-the-cell}

There are many ways of transferring energy. To heat a house in the
western world, we most often have a centralized heating (and sometimes
cooling) system, where the heat is generated, e.g., in a furnace, and
then transferred to the rest of the house via pipes and the like. The
equivalent system might be for mammals the blood that gets re-oxygenated
with the lungs, before it delivers oxygen throughout the body.

At the cellular level, however, \protect\hyperlink{ATP}{ATP} or
Adenosine TriPhosphate is created throughout the cell near the
equivalent of the little furnaces: mitochondria. Needless to say, there
is no combustion with the mitochondria, and yet transfer of electrons
and energy delivered. The energy is transferred from the
\protect\hyperlink{OM}{organic matter} to ATP, and ATP being a
relatively small molecule, can easily reach all metabolic processes,
usually operated by proteins, which need energy to proceed (to overcome
the \href{https://en.wikipedia.org/wiki/Activation_energy}{activation
energy} mentioned above).

\begin{figure}

{\centering \includegraphics[width=0.7\linewidth]{pictures/atp02a} 

}

\caption{ATP molecular structure containing adenosine (= adenine + ribose) and the three phosphate linked together with a pyrophosphate energy rich bond. Copyrighht Pearson Benjamin Cummings}\label{fig:ATPstructure}
\end{figure}

The key to storing energy is in the bond between the phosphate, which
are often referred to as high-energy bonds. High-energy phosphate bonds
are pyrophosphate bonds, acid anhydride linkages formed by taking
phosphoric acid derivatives and dehydrating them. As a consequence, the
hydrolysis of these bonds is exergonic under physiological conditions,
releasing energy. As ATP releases its energy by the break of the
pyrophosphate bond, ATP thus becomes ADP (Adenosine Diphosphate) and
releases a phosphate in the meantime, like in figure \ref{fig:ATPtoADP}.
A common representation of this reaction is
\(ATP \Rightarrow ADP + P_i\) with the P\textsubscript{i} referring to
as a phosphate.

\begin{figure}

{\centering \includegraphics[width=0.8\linewidth]{pictures/atptoadp} 

}

\caption{Release of energy ATP hydrolyzed into ADP. http://hyperphysics.phy-astr.gsu.edu/hbase/Biology/atp.html}\label{fig:ATPtoADP}
\end{figure}

Now that we have allocated electrons on atoms and molecules, you easily
understand that the number of electrons on the outside of the triplet
\(PO_3-O-PO_2-O-PO_2\) is huge. In his fantastic book, Degens
\citeyearpar{Degens1989-ip} suggests that

\begin{itemize}
\tightlist
\item
  because of the plethora of electrons on the triphosphate part of ATP,
\item
  because the tetrahedral phosphate molecules are linked together by one
  of the corners of each tetrahedron (figure \ref{fig:PO4tetrahedra}),
\item
  and because of the \(\pi\) electron bonds (double bond) which tends to
  repulse each other \citep{Degens1989-ip},
\end{itemize}

the triphosphate molecule can only be in constant movement, which
`maintains the animated world' \citep{Degens1989-ip}.

\begin{figure}

{\centering \includegraphics[width=0.6\linewidth]{pictures/204phosphate} 

}

\caption{Tetrahedral configuration of each of the phosphate molecule which when linked together supposedly are in constant movement}\label{fig:PO4tetrahedra}
\end{figure}

\section{The ATP manufacture}\label{the-atp-manufacture}

If ATP freely releases energy for a reaction to proceed in the cell, as
ATP becomes ADP + Pi, there must be energy at one point to `manufacture'
or recharge the ATP molecule in the first place. We have seen that much
of the energy in the cell is transferred thanks to the ATP molecule. So
could we just make ATP from the energy liberated by ATP??? The answer is
no, obviously\ldots{} The energy to create ATP has to come from
elsewhere, otherwise it just cannot work\ldots{}!!!

Life has found a wonderful way to do that thanks to proton flows. Yes,
that is correct, a proton flow. ATP is literally formed from an ADP and
a phosphate and assembled thanks to a protein called ATP synthase, or
ATPase, which seems to be activated or powered almost mechanically, very
much like a water mill, by a proton flow (Figure \ref{fig:ATPase}). The
`head' of the ATPase, which looks a bit like a `mushroom hat', referred
to as the F\textsubscript{O} region, rotates as a result of the proton
flow and allows the phosphorylation of ADP. It is thus fair to call the
ATPase a `proton mill'. You may want to look at this very nice concise
document on \href{https://en.wikipedia.org/wiki/ATP_synthase}{ATP
synthase on Wikipedia}, which provides more detailed information and
shows quite nicely how the mechanical force of the H\textsuperscript{+}
flow is thought to allow P\textsubscript{i} to be attached to an ADP.
This part of the energy recovery is referred to as
\emph{\textbf{phosphorylation}}.

\begin{figure}

{\centering \includegraphics[width=0.4\linewidth]{pictures/Atpsyntase4} 

}

\caption{Artist representation of the ATP synthase powered, almost mechanically by proton flows. By The original uploader was Asw-hamburg at German Wikipedia - Transferred from de.wikipedia to Commons by Leyo using CommonsHelper., CC BY-SA 3.0, https://commons.wikimedia.org/w/index.php?curid=8993938}\label{fig:ATPase}
\end{figure}

So, this mechanical force is what generates the synthesis of ATP from
ADP, or again \emph{\textbf{phosphorylation}}. So this solves that part
of the ATP synthesis. Now, another mechanism must be responsible for the
formation of the proton flow.

\section{Creating a proton gradient as a source of proton
flow}\label{creating-a-proton-gradient-as-a-source-of-proton-flow}

Until now, I have not even mentioned where all this takes place. We are
about to see where this happens. But first, let us take an analogy to
better understand the proton flow responsible for the synthesis of ATP.
Flow of matter just does not happen on its own, it only happens as a
result of a \emph{\textbf{gradient}} between a compartment and another.
Using the water mill analogy, the reason why there is release of energy
in the form of kinetic energy of the water is because there is a drop in
elevation, and therefore in potential energy between the `compartments'
upstream and downstream the mill. To increase the hydraulic gradient,
men throughout the world have built dams to create compartments and
increase the hydraulic gradient between upstream and downstream the
mill. So in the water mill analogy, the hydraulic gradient is maintained
thanks to the dam which creates compartments, and to the continuous
arrival of new water upstream and to the leaving of water downstream. So
three things are necessary for the release of kinetic energy (which in
the end is going to power the rotation of a mill wheel, and in the old
days would grind wheat or corn grain):

\begin{itemize}
\tightlist
\item
  a system to compartmentalize water,
\item
  a supply of water, and,
\item
  an outlet for the water so that is does not accumulate.
\end{itemize}

By analogy, the formation of ATP corresponds to the grain grinding; the
proton flow corresponds to the water flowing that activates the mill
wheel, and in our case the proton flow activates the ATPase. We now need
to address the three parts of the system needed to have a proton
gradient and flow:

\begin{itemize}
\tightlist
\item
  a compartment system,
\item
  a proton supply system, and,
\item
  a system to use protons so the gradient is maintained.
\end{itemize}

\subsection{A compartment to accumulate
protons}\label{a-compartment-to-accumulate-protons}

We saw earlier that the \protect\hyperlink{phospholipids}{double
phospholipid layer} made the cell membrane. Well if you close a membrane
on itself you will make a compartment or a balloon. It turns out that if
you look at a typical prokaryotic cell and at organelles in a eukaryotic
cell (Figure \ref{fig:cells}), there are compartments formed between a
cell wall and the plasma membrane for prokaryotic cells, and between two
double layer phospholipid membranes for the eukaryotic cells. So in
reality, it is a bit as if one were to create a space between two
balloons.

\begin{figure}

{\centering \includegraphics[width=0.45\linewidth]{pictures/prokaryotic-cell-diagram} \includegraphics[width=0.45\linewidth]{pictures/Mitochondrion_mini.svg} 

}

\caption{Artist representation of a prokaryotic cell and a mitochondrion from a eukaryotic showing the intermembrane spaces: between the cell wall and plasma membrane for the prokaryotic cell, and, between the outer and inner membranes for the mitochondrion.  https://micro.magnet.fsu.edu/cells/animals/animalmodel.html and By Kelvinsong - Own work, CC0, https://commons.wikimedia.org/w/index.php?curid=27715320}\label{fig:cells}
\end{figure}

At the end though, there exists for prokaryotic cells and mitochondria
(and most organelles for eukaryotic cell) an inter-membrane space, which
effectively creates compartments to accumulate `things'. The double
phospholipid membrane is so strong and so tight that it is
\emph{proton-tight}. In other words, if somehow protons accumulate in
this inter-membrane space, protons cannot leave through the membrane,
they can only leave through designated areas. And yes, you guessed
right, the designated area for protons to leave the inter-membrane space
is the ATP synthase! So to take another analogy, you can imagine an
inner tube full of air under pressure, air can only leave through the
valve. And if you were to put a little fan in front of the valve as you
are releasing air, the fan would rotate. Imagine that this is
essentially what happens: the ATPase is both the valve and the fan as it
\href{https://upload.wikimedia.org/wikipedia/commons/6/62/ATPsyn.gif}{literally
rotates with the proton flow}.

\subsection{A supply of protons for the intermembrane space: proton
pumps}\label{a-supply-of-protons-for-the-intermembrane-space-proton-pumps}

Now, the next natural question is that in the case of an inner tube,
somebody pumped some air into the tube to put it under pressure.
Similarly, there must be a system that pumps protons into the
inter-membrane space. And yes indeed, you guessed it right, there are
proton pumps that do the job. But again, these pumps must be powered by
some sort of energy.

Let us pause for a second. We have seen in section
\ref{generating-energy-transfer-of-electrons} that generally speaking,
energy is generated from the transfer of electrons. But until now, we
have not even mentioned electrons: only ATP and protons\ldots{}? This is
where electrons come in: \textbf{energetic electrons transported onto
electron transfer molecules are the ones that power the proton pumps!}.
It is time to take a bird's eye view again. Energy liberation exists
when electrons are transferred from an electron donor to an electron
acceptor. Life has found a way to capture this energy in the form of
ATP, which allows transportation of energy to the needed places in the
cell. So the energy transfer in the cell does not have to be totally
instantaneous like combustion would be, and it can be spent exactly
where it needs to be spent.

Yet, the production of ATP is in fact generated by an electron transfer,
but indirectly.

\begin{itemize}
\tightlist
\item
  No, electrons are \textbf{NOT} transferred from organic matter to ATP.
\item
  Yes, electrons are \emph{ultimately} transferred from organic matter
  to an electron acceptor, which in the case of aerobic respiration is
  O\textsubscript{2}.
\end{itemize}

But just like the generation of ATP is indirectly linked to the electron
transfer, the transfer of electrons from organic matter to the ultimate
electron acceptor is also indirect. A set of molecules called
\textbf{electron transfer molecules} are intermediate carrier of
electrons and are the ones which power the proton pumps as represented
in figure \ref{fig:protonpumps} below.

\begin{figure}

{\centering \includegraphics[width=1\linewidth]{pictures/membrane_proton_pumps} 

}

\caption{Artist representation of the ATP synthesis powered by a proton flow, itself powered by a proton gradient, itself produced thanks to proton pumps powered by energy rich electrons carried by electron transfer molecules. Obtained from 2006 Pearson Education}\label{fig:protonpumps}
\end{figure}

\hypertarget{electron-transfer-molecules-that-power-the-proton-pumps}{\subsection{Electron
transfer molecules that power the proton
pumps}\label{electron-transfer-molecules-that-power-the-proton-pumps}}

The two main electron transfer molecules in the respiration process are
called NAD (Nicotinamide Adenine Dinucleotide; figure \ref{fig:NAD}) and
FAD (flavine adenine dinucleotide; figure \ref{fig:FAD}). These two
nucleotides have the ability to be reduced (= gain electrons) and
oxidized (= lose electrons), and because they are mobile, to carry
electrons from the cell (bacteria) or mitochondrion (eukaryot) cytoplasm
to the proton pumps.

\begin{figure}

{\centering \includegraphics[width=0.45\linewidth]{pictures/NADreduction} \includegraphics[width=0.45\linewidth]{pictures/NADfull} 

}

\caption{Molecular formula of Nicotinamide Adenine Dinucleotide (NAD) in both oxydized and reduced states}\label{fig:NAD}
\end{figure}

\begin{figure}

{\centering \includegraphics[width=1\linewidth]{pictures/FAD} 

}

\caption{Molecular formula of Flavine Adenine Dinucleotide (FAD) in both oxydized and reduced states}\label{fig:FAD}
\end{figure}

\section{Transfer of electrons from the Organic Carbon to an electron
acceptor}\label{transfer-of-electrons-from-the-organic-carbon-to-an-electron-acceptor}

So this answers the question of how the proton pumps are powered, but it
is now time to look at the global fate of electrons: where do they come
from and where do they end up? Actually, we already know the global
answers to these questions: the electrons are stored unto the C, N, and
S of organic molecules and they are eventually accepted by an electron
acceptor; in the case of aerobic respiration, O\textsubscript{2} is the
ultimate electron acceptor.

Before there might be any confusion, in the aerobic respiration process
that involves organic molecules as electron donors (we will see that
inorganic molecules can also be electron donors), strangely enough, the
only atom which donates electron is \textbf{carbon} while the nitrogen
and sulfur atoms keep all their eight electrons. Specialized microbes,
called \emph{\protect\hyperlink{lithotrophs}{lithotrophs}}, literally
`feed on stone', which means that their source of electron is from an
inorganic molecule, have the ability to use ammonium and hydrogen
sulfide as electron donor (8 electrons available for `donation') in
their aerobic respiration chains. So all this to say that in aerobic
respiration that involves organic matter, \textbf{only} the carbon atom
donates its electrons. The organisms which use organic matter as
electron donors are called \emph{organotrophs}. We, as mammals and
humans, are \emph{organotrophs}.

\subsection{Oxygen reduction}\label{oxygen-reduction}

Let us start from the end of the process: the reduction of oxygen into
H\textsubscript{2}O, as represented on figure \ref{fig:protonpumps}
above. As the NAD and FAD are powering proton pumps (which names include
FMN, ubiquinone, cytochrome a, b, and c see figure
\ref{fig:protonpumps}), the electrons lose some of their energy and
migrate within the phospholipid membrane from the proton pumps towards
the ATPase. Because of the involvement of many different proteins
involved in this transport, this is referred to as the
\emph{\textbf{electron transfer chain}}. Because the NAD and the FAD
molecules are oxidized (they have given up their electrons), and because
at the ATPase location, ADP gains a phosphate to become ATP, the whole
process is referred to as \emph{\textbf{oxidative phosphorylation}}. At
the confluence between the ATPase and the cytochrome a\textsubscript{3},
the reaction \eqref{eq:O2reduction} occurs:

\begin{equation}
O_2 + 4 e^- + 4 H^+ \Rightarrow 2 H_2O
\label{eq:O2reduction}
\end{equation}

This is the place where the electrons are accepted!!! So now you know
exactly where all this is happening! Is not that wonderful? So the
electrons are accepted by O\textsubscript{2} (remember on a molecule of
O\textsubscript{2}, each Oxygen atom only has 6 electrons for itself, so
each can accept two more), and these electrons reduce oxygen into
H\textsubscript{2}O. This has several consequences:

\begin{itemize}
\tightlist
\item
  the reduction of O\textsubscript{2} also consumes 4
  H\textsuperscript{+}, which solves the necessity for an outlet for
  protons as they flow out of the inter-membrane through the ATPase,
  which was the third condition to maintain a proton gradient
\item
  the reduction of O\textsubscript{2} obviously consumes electrons,
  which also provides an outlet for the electrons. In order words, if no
  electron acceptors are available, then the electrons at the proton
  pumps have no outlet, so the proton pumps are stalled, which in turn
  halts the maintaining of a proton gradient, which eventually stops the
  production of ATP.
\end{itemize}

\subsection{Electron fate from organic matter to electron transfer
molecules}\label{electron-fate-from-organic-matter-to-electron-transfer-molecules}

The fate of electrons from organic matter or glucose to the electron
transfer molecules involves two processes called
\protect\hyperlink{glycolysis}{glycolysis} and the
\protect\hyperlink{krebscycle}{Krebs or citric acid cycle}. Biology
majors have to learn in details all the steps and the name of the
molecules involved in these chain reactions. This is beside our point
for our class. Instead I want you to understand and know that these
reactions involve

\begin{itemize}
\tightlist
\item
  transfer of electrons from organic carbon onto
  \protect\hyperlink{electron-transfer-molecules-that-power-the-proton-pumps}{electron
  transfer molecules}
\item
  the release of carbon atoms which have lost all their electrons,
  therefore releasing C in the form of CO\textsubscript{2}.
\item
  the hydrolysis of glucose molecules to
  \href{https://en.wikipedia.org/wiki/Pyruvic_acid}{pyruvate}, a C3
  molecule (a 3 carbon organic molecule) during the glycolysis, which
  then loses CO\textsubscript{2} as it enters the citric cycle
\item
  the acetyl-coA, a co-enzyme is key to incorporate organic carbon in
  the Krebs cycle.
\end{itemize}

In figure \ref{fig:respirationglobal} below which summarizes all the
aerobic respiration pathways, you can see that most of the ATP is formed
thanks to the proton flow, and that
\protect\hyperlink{electron-transfer-molecules-that-power-the-proton-pumps}{electron
transfer molecules} are involved in all steps.

\begin{figure}

{\centering \includegraphics[width=0.95\linewidth]{pictures/respiration_global} 

}

\caption{Summary of cellular respiration with the ATP and electron transfer budget. Obtained from https://cdn.thinglink.me/api/image/847806852426104839/1240/10/scaletowidth}\label{fig:respirationglobal}
\end{figure}

This \ref{fig:respirationglobal} diagram does not show where the
CO\textsubscript{2} are produced so I invite to see, in addition to the
notes given in class, that CO\textsubscript{2} is released as pyruvic
acid enters in the figures \ref{fig:krebs} and \ref{fig:glycolysis}
below.

\begin{figure}

{\centering \includegraphics[width=0.95\linewidth]{pictures/Glycolysis} 

}

\caption{Glycolysis pathway. Image}\label{fig:glycolysis}
\end{figure}

\begin{figure}

{\centering \includegraphics[width=0.95\linewidth]{pictures/Krebscycle} 

}

\caption{Krebs or citric acid cycle pathway.}\label{fig:krebs}
\end{figure}

\section{Summary of respiration}\label{summary-of-respiration}

We have purposely presented the molecular processes of respiration
starting from the end and moving back to the beginning of the
respiration chain. We can summarize the respiration process again in
that order:

\begin{itemize}
\tightlist
\item
  the goal of respiration is to transfer energy initially stored as high
  energy electrons on organic molecules to ATP, the energy currency of
  the cell
\item
  The production of ATP itself has to come from a different source of
  energy than that liberated by ATP itself
\item
  ATP is synthesized by the phosphorylation of ADP by the ATP synthase,
  itself powered by a proton flow from the inter-membrane space to the
  cytoplasm space
\item
  the proton flow is powered by a proton gradient between the
  inter-membrane space and the cytoplasm space
\item
  This gradient is made possible thanks to

  \begin{itemize}
  \tightlist
  \item
    a proton-tight compartment corresponding to the inter-membrane space
  \item
    a supply of protons from the cytoplasm to the inter-membrane space,
    fed by proton pumps
  \item
    an outlet for the protons flowing out and powering the ATPase (`the
    proton mill'), which combined with the reduction of oxygen form
    water molecules
  \end{itemize}
\item
  The proton pumps are themselves powered by the oxidation of electron
  transfer molecules (NAD and FAD), which carry high energy electrons
  from the Krebs or citric cycle to the electron transfer chain
\item
  The electron transfer molecules accept the electrons, or are reduced
  during glycolysis, before and during the Krebs cycle as organic carbon
  are oxidized or lose their electrons and release CO\textsubscript{2}.
\end{itemize}

The other way to present respiration is to say that:

\begin{itemize}
\tightlist
\item
  Energy rich electrons stored on organic molecules are transferred onto
  electron transfer molecules during glycolysis, before and during the
  Krebs cycle
\item
  The high energy electrons thus transported power proton pumps placed
  in the inner membrane of the cell for unicellular organisms or of the
  mitochodrion for eukaryotic cells
\item
  These pumps feed a supply of protons to the inter-membrane space
  which, because this space is proton-tight, and because protons outlets
  are limited to the ATP synthase, creates an accumulation of protons in
  this space
\item
  The accumulation of protons creates a proton gradient between the
  inter-membrane space and the cytoplasm, which generates a proton flow
  at designated `holes' in the membrane
\item
  The protons flow out the inter-membrane through the ATP synthase which
  can be approximated by proton canals and almost mechanically turn the
  ATPase head which acts as a `proton mill'
\item
  This proton mill catalyzes the phosphorylation of ADP into ATP
\item
  The proton gradient is maintained possible because the protons flowing
  out of the inter-membrane space are combined with electrons reducing
  oxygen into water
\item
  The electron flow is maintained thanks to the same oxygen reduction,
  which in turns allows electron transfer molecules to be oxidized so
  that then can take their `proton load' again.
\item
  Overall respiration consists in transferring high energy electrons
  from organic molecules to oxygen, and by doing so creating a proton
  gradient which in turn is the main driver for the formation of ATP in
  the cell.
\end{itemize}

Both summaries present the same story but have their own logic. Other
important subtleties of respiration include:

\begin{itemize}
\tightlist
\item
  In the case organic molecules are the source of electrons, only the
  carbon atom donates electrons, the amine and thiol radicals being
  eliminated as a by-product DO NOT donate their electrons in this
  respiration system. This is admittedly a bit weird as the the amine
  and thiol groups still have 8 electrons to donate each.
\end{itemize}

\section{Respiration electron flow
schemes}\label{respiration-electron-flow-schemes}

\subsection{Linkages between respiration and microbial
processes}\label{linkages-between-respiration-and-microbial-processes}

These few paragraphs and pages were necessary but we now need to connect
them with some of the practical consequences that concern ecological
engineering.

\begin{itemize}
\tightlist
\item
  The electron donors are the fuel for all microbial processes involved
  in the substrate of `treatment systems'.
\item
  The byproduct of the loss of electrons originally stored on the
  electron donor will be a source of inorganic molecules that matter
\item
  When the electron acceptor is O\textsubscript{2}, this results in an
  oxygen demand which may have consequences on the overall oxygen level
  and may lead to anaerobic conditions, where other electron acceptors
  come in play
\item
  We try to take advantage of the anaerobic conditions to dissipate or
  strip excess nutrients, e.g.~NO\textsubscript{3}\textsuperscript{-}
\end{itemize}

\subsection{Respiration schemes}\label{respiration-schemes}

So we thought we could provide a nice concise summary of respiration to
highlight the points that matter for environmental and microbial
considerations in the form of \emph{respiration schemes}, still
maintaining some of the important drivers of the formation of ATP. In
the first respiration scheme in Figure \ref{fig:generic-resp-scheme}
below, the transfer of electrons is represented by the \emph{blue wavy
line}. Electrons are transferred from an electron donor to an electron
acceptor. The consequences of the transfer are represented in
\emph{thicker light blue arrows}. The first consequence of the transfer
of electrons is the powering of proton pumps, which in turn create a
proton gradient in the inter-membrane space, which result into proton
flow, which gives the necessary energy to catalyze the phosphorylation
of ADP into ATP. The products of oxidation, reduction, and
phosphorylation are represented in black arrows.

\begin{figure}

{\centering \includegraphics[width=0.75\linewidth]{pictures/respiration-generic} 

}

\caption{Generic respiration scheme illustrating the flow of electrons from an electron donor to and electron acceptor, the by-products of the respective oxidation and reduction, and the consequences on proton gradients and flow, which ultimately help catalyse the formation of ATP}\label{fig:generic-resp-scheme}
\end{figure}

The respiration scheme is obviously an oversimplified representation of
the what happens as described above, but generally holds true. Some ATP
are generated during glycolysis and in association with the Krebs cycle,
but represent 4 out of a total of 38, or about 10\% of the total. So it
is probably fair to simplify things to represent the majority of the
processes (here 90\%). The beauty of this scheme is that it applies to
just about all types of electron donors and acceptors.

\subsection{Aerobic respiration schemes for
organotrophs}\label{aerobic-respiration-schemes-for-organotrophs}

In the next scheme, we take the classic example of aerobic respiration,
where glucose is considered as the electron donor, and
O\textsubscript{2} the electron acceptor.

\begin{figure}

{\centering \includegraphics[width=0.75\linewidth]{pictures/respiration-glucose-O2} 

}

\caption{Aerobic respiration scheme illustrating the flow of electrons from glucose to O~2~, the by-products of the respective oxidation and reduction being CO~2~ and H~2~O, and H~2~O. The consequences of the transfer of electrons on proton pumps, gradients and flow, which ultimately help catalyse the formation of ATP, do not change from the generic respiration scheme above}\label{fig:aerobic-resp-scheme-glucose}
\end{figure}

Using the generic carbohydrate CH\textsubscript{2}O as a source of
electrons, it is possible to express the donation of electrons in
equation \eqref{eq:OM-edonor} (redox half reaction):

\begin{equation}
CH_2O + H_2O \rightarrow CO_2 + 4 H^+ + 4 e^- 
\label{eq:OM-edonor}
\end{equation}

Organisms which use \emph{organic} molecules as electron donors are
referred to as \protect\hyperlink{trophic-names}{organotrophs}, hence
the title. We shall see later that some microorganisms are able to use
inorganic molecules as their electron donors. In our case (Figure
\ref{fig:aerobic-resp-scheme-glucose}), the byproduct of glucose as the
carbon atoms lose all their electrons becomes CO\textsubscript{2} as
show in equation \eqref{eq:OM-edonor}.

In turn, the electrons are accepted by O\textsubscript{2} following
equation \eqref{eq:O2-eacceptor}

\begin{equation}
O_2 + 4 H^+ + 4 e^- \rightarrow 4H_2O  
\label{eq:O2-eacceptor}
\end{equation}

The combination of donating electrons from glucose (equation
\eqref{eq:OM-edonor}), or organic matter illustrated in Figure
\ref{fig:aerobic-resp-scheme-OM}, combined with the accepting of these
electrons by O\textsubscript{2} (equation \eqref{eq:O2-eacceptor}),
results in the combined overall summary of aerobic respiration:

\begin{equation}
 CH_2O + O_2 \rightarrow CO_2 + H_2O  
\label{eq:sum-resp-OM-O2}
\end{equation}

The third scheme (Figure \ref{fig:aerobic-resp-scheme-OM}), the generic
term organic matter is used to show that in reality, not only glucose
can be part of the electron transfer chain, but essentially all types of
organic molecules, as represented in Figure \ref{fig:respiration-OM}. So
the only difference with the previous scheme with glucose only, are the
by-products of the \protect\hyperlink{catabolism}{catabolism} of organic
matter: CO\textsubscript{2}, NH\textsubscript{4}\textsuperscript{+},
HS\textsuperscript{-}, and PO\textsubscript{4}\textsuperscript{3-}.

\begin{figure}

{\centering \includegraphics[width=0.75\linewidth]{pictures/respiration-OM-O2} 

}

\caption{Aerobic respiration scheme illustrating the flow of electrons from organic matter to O~2~, the by-products of the oxidation of OM include ammonium, hydrogen sulfide and phosphate this time}\label{fig:aerobic-resp-scheme-OM}
\end{figure}

The important thing to notice here is that the N and S atoms, which in
\emph{organic molecules} possess 8 electrons for themselves, or are
fully reduced, \textbf{stay reduced} as byproducts of respiration. This
might appear odd as mentioned before, because there are electrons
available for another oxidation to take place.
\protect\hyperlink{trophic-names}{Lithotroph micro-organisms} are
specialized in taking advantage of these available electrons, as we
shall see later. The important message here is that you can see that
respiration processes essentially breakdown large molecules into small
inorganic ones, and that in the case of respiration for
\protect\hyperlink{trophic-names}{organotrophs}, the only atom to lose
its electrons is carbon.

\begin{figure}

{\centering \includegraphics[width=0.75\linewidth]{pictures/respiration-3-OM-families} 

}

\caption{Oxidative phosphorylation of organic molecules of carbohydrates but of other molecular families as well. Figure obtained from Essential Cell Biology from Garland Science}\label{fig:respiration-OM}
\end{figure}

If we think about this apparent oddity, N and S are a lot more
electronegative than C, so in the organotroph cell, it is a lot easier
for oxygen to strip electrons from carbon than it is from N and S. It
might be worth taking a hydraulic analogy: in a soil, pore water will
tend to drain through the area of a soil that has the highest hydraulic
conductivity, or use the path of least resistance, totally by-passing
areas of lower hydraulic conductivity. Chemically, it is just a lot
easier to obtain electrons and energy out of the reduced carbon, and a
lot harder to obtain them from amine (-NH\textsubscript{2}) and thiol
(-SH) radicals; this is why N and S stay in a reduced form as byproducts
of \protect\hyperlink{trophic-names}{organotrophic} respiration. Again,
several group of \protect\hyperlink{trophic-names}{lithotrophs}
specifically target the electrons on ammonium and on Hydrogen sulfide,
and use O\textsubscript{2} as their electron acceptor.

In the next chapter, we will introduce all the respiration processes,
aerobic and anaerobic, and performed by organotrophs and lithotrophs.

\chapter{The classic redox sequence of wetland
soils}\label{the-classic-redox-sequence-of-wetland-soils}

In the previous chapter, we introduced quite a few details on the
molecular functioning of respiratory processes. We took aerobic
respiration as our model to determine that most of the ATP produced in
the cell is principally due to a proton flow from the inter-membrane
space to the cytoplasm, for microbial cells and within the mitochondria
for eukaryotic cells (Figure \ref{fig:respirationglobal}). The
respiration schemes provide a very handy method to summarize the
important drivers and consequences of respiration: the electron donors,
the electron acceptors, and the byproducts of both oxidation and
reduction. In this chapter we will use the same respiration schemes to
explore the processes at play in wetland soils and the consequences on
concentration gradients and movement of molecules of importance in
environmental and ecological engineering.

\section{The theoretical vertical sequence of respiratory processes in
wetland
soils}\label{the-theoretical-vertical-sequence-of-respiratory-processes-in-wetland-soils}

For this we will use a theoretical wetland soil to illustrate our point.
Such theoretical soil has `enough' organic matter content for microbial
respiration to take place throughout its profile, `enough' of other
sand, silt, and clay and all the minerals that accompany them, including
iron and manganese oxides. Let us assume that this theoretical soil, is
sufficiently moist and aerated for microbial aerobic respiration to take
place throughout the soil profile. Let us then assume that this soil is
suddenly flooded, and let us explore the consequences of this.
Respiratory processes are going to occur on a temporal sequence, which
in fact will be mirrored by vertical sequence of processes in the
wetland sediment.

\section{An aerobic layer near the soil-water
interface}\label{an-aerobic-layer-near-the-soil-water-interface}

At first, all micro-organisms are going to use all the oxygen that might
be present in the pore space. But because pore space is now getting
filled with water, it is possible that the amount of oxygen available
for microorganisms might change. And yes, indeed, water does not have
nearly the capacity to provide oxygen as air does for several reasons:

\begin{itemize}
\tightlist
\item
  at 15°C in water there are about 10 mg of O\textsubscript{2} in one
  liter of water. Comparatively, in one liter of air, there is about 300
  mg of O\textsubscript{2}, or \textbf{30 times} more. How does one
  calculate this? At standard conditions, 1 liter of air at 21\% oxygen
  possesses 0.21 L of oxygen. Since for these conditions, 1 mole of gas
  occupies 22.4 L, simply divide 0.21/22.4, to arrive at 0.0094 moles of
  oxygen. Then the mass of oxygen in 1 liter of air is 0.0094×32 g/mole
  = 300 mg.
\item
  the diffusivity (which quantifies the ability of elements to move
  about) of O\textsubscript{2} in air is 0.176 cm²/s while that of
  O\textsubscript{2} in water is 2.10×10\textsuperscript{−5} cm²/s, or
  more than \textbf{8,000 times smaller}
  \citep{Wikipedia_contributors2017-id}
\end{itemize}

In other words, it is good to remember that \textbf{the amount of oxygen
available in water compared to air is about 30 less, and that oxygen in
water moves almost 10,000 slower than in water}.

So, one can clearly see that the potential supply of oxygen from flooded
porewater is thus a lot more limited than in aerated pore space. Now,
where is the potential source for supply of oxygen for our recently
flooded soil? The answer is the atmosphere. For a real flooded wetland
soil, another source of O\textsubscript{2} might be photosynthesis from
algae and aquatic vegetation during the day. But in all cases, most of
the oxygen needed for microbial aerobic respiration of our new flooded
soil will have to travel the distance corresponding to the thickness of
the water column, and, the linear distance within the porous medium of
the soil, which can be quite \emph{tortuous}. This property of the
soil/sediment has been factored in by researchers and been called
\emph{tortuosity}.

\textbf{In summary}, the supply of oxygen to bacteria in a flooded soil
is limited because of four factors:

\begin{enumerate}
\def\labelenumi{\arabic{enumi}.}
\tightlist
\item
  there is about 30 times less O\textsubscript{2} in water than there is
  in the same volume of air
\item
  the O\textsubscript{2} diffusive transport capacity in water is about
  10,000 times smaller than that in air
\item
  O\textsubscript{2} has to travel from the atmosphere to the soil
  through the thickness of the water column, and
\item
  travel through the tortuous path of the soil porous medium
\end{enumerate}

In addition to these four rules, which apply to all aquatic systems, the
velocity of water matters very much in the reaeration process, which
corresponds to factor 1 above: reaeration is much higher for streams
than for stagnant waters. In other words, the stagnant water above our
flooded wetland soil example, is another factor, compared to streams,
which further limits the supply of oxygen to the aerobic bacteria in the
flooded soil.

Not surprisingly, this supply is just too limited compared to the
demand. As a result, most of our recently flooded soil bacteria consumes
all the O\textsubscript{2} and the only part of the sediment that might
have a little bit of oxygen is the area at the soil-water interface.
This is what it illustrated in Figure
\ref{fig:aerobic-layer-wetland-soil} below.

\label{fig:aerobic-layer-wetland-soil}Animation summarizing the formation of
aerobic and anaerobic layers of a theoretical flooded wetland soil due
to the imbalance between bacterial respiratory oxygen demand and oxygen
supply through the water column

Because the O\textsubscript{2} demand exceeds the O\textsubscript{2}
supply from the atmosphere and the water column, an O\textsubscript{2}
concentration gradient forms from the soil-water interface down. This
concentration gradient, in turn, drives a downward movement of oxygen
from the water column into the sediment down to the depth where there is
no more concentration gradient. This depth defines the beginning of the
anaerobic zone of the sediment, and above, the aerobic layer of the
sediment.

\section{Respiration in the anaerobic zone of the
soil}\label{respiration-in-the-anaerobic-zone-of-the-soil}

What happens to all the microbes in the anaerobic zone of the sediment?
Certainly the exclusively aerobic microbes just do not survive, but most
bacteria are facultative aerobs. In other words, they can switch from
aerobic to anaerobic respiration. Let us state this again: in the
anaerobic zone of the sediment, only \textbf{unicellular microorganisms
are able to survive} and have had to adapt their respiration to still be
able to produce ATP for their metabolism, but using electron acceptors
\emph{other than O\textsubscript{2}}.

It turns out that thermodynamics dictate that not all electron acceptors
can generate the same amount of energy when they strip electrons from
their electron donors. As a result, one can classify electron acceptors
in decreasing order from the most to the least oxidizing, and the list
of preferred electron acceptors goes as such:

\begin{itemize}
\tightlist
\item
  nitrate or NO\textsubscript{3}\textsuperscript{-}
\item
  Manganese oxide (MnO\textsubscript{2}) or Mn IV
\item
  Iron oxides/hydroxyde or Fe III
\item
  sulfate or SO\textsubscript{4}\textsuperscript{2-}
\item
  Carbon dioxide or CO\textsubscript{2}
\end{itemize}

Although very different microbes are involved at the different stages,
the apparent \emph{demand} for electron acceptor, in our theoretical
wetland soil, can be described as a \emph{temporal} sequence of events:
oxygen is the preferred electron acceptor; when O\textsubscript{2} is
all used, NO\textsubscript{3}\textsuperscript{-} will be used as the
preferred electron acceptor; when all the
NO\textsubscript{3}\textsuperscript{-} is used, the next most powerful
electron acceptor is MnO\textsubscript{2} (also referred to Mn IV; IV
corresponds to its \protect\hyperlink{oxidation-state}{oxidation
state}), which is present in a solid or mineral form in soils; when all
the MnO\textsubscript{2} is used, then iron oxide or hydroxyde, which
are also in the solid phase (also referred to Fe III, III corresponds to
its \protect\hyperlink{oxidation-state}{oxidation state}) will be used
as the preferred electron acceptor; when all the Fe is used, then
SO\textsubscript{4}\textsuperscript{2-} is used as the next preferred
electron acceptor, and then when finally all the other electron
acceptors have been all used, CO\textsubscript{2} can be the ultimate
electron acceptor\ldots{}!

Finally, to this temporal sequence corresponds a theoretical
\emph{spatial} sequence or layers where each of the electron acceptor
essentially defines a soil layer, with the layers are organized with
depth from the most to the least oxidizing electron acceptor as
represented in Figure \ref{fig:wetland-soil-layers}

\begin{figure}

{\centering \includegraphics[width=0.55\linewidth]{pictures/wetland-soil-layers} 

}

\caption{Theoretical spatial layering of wetland soils corresponding to the electron acceptor available, *not too long after flooding*. In each layer, the oxidizing and the reduced forms are illustrated as oxidizing/reduced. *Not to scale*}\label{fig:wetland-soil-layers}
\end{figure}

It is now time to present the respiration processes in each of the
\emph{redox} layer.

\hypertarget{denit}{\section{A denitrification layer below the aerobic
layer}\label{denit}}

The next most powerful or oxidizing electron acceptor after
O\textsubscript{2} is NO\textsubscript{3}\textsuperscript{-}. We have
seen in the previous chapters that the N atom in nitrate has zero
electron for itself, hence its ability to accept electrons. Just like
for aerobic respiration, nitrate reduction just does not happen on its
own. Facultative anaerobic bacteria called \emph{denitrifiers} take
advantage of the electrons available on organic matter and of nitrate to
accept them to generate their energy. The denitrification of this
theoretical wetland soil is referred to as
\protect\hyperlink{trophic-names}{heterotrophic denitrification} because
the source of carbon for these denitrifiers is also the source of
electron, as OM is the source of both. On a side note, there are
\protect\hyperlink{trophic-names}{autolithotrophic denitrifiers}, which
use \href{https://en.wikipedia.org/wiki/Pyrite}{pyrite}
FeS\textsubscript{2} as their source of electrons, and have to find
their carbon from another source than OM.

There are two possible byproducts of nitrate reduction:
N\textsubscript{2}, which is the inert gas that makes 78\% of our
atmosphere, and \protect\hyperlink{nitrous-oxide}{N\textsubscript{2}O},
which is a potent \protect\hyperlink{GHG}{greenhouse gas}. The byproduct
of the oxidation of OM are the same as the ones in aerobic heterotrophic
respiration, i.e., CO\textsubscript{2},
NH\textsubscript{4}\textsuperscript{+}, HS\textsuperscript{-}, and
PO\textsubscript{4}\textsuperscript{3-}.

\begin{figure}

{\centering \includegraphics[width=0.75\linewidth]{pictures/respiration-OM-NO3} 

}

\caption{Respiration scheme for heterotrophic denitrification}\label{fig:denit-resp}
\end{figure}

In reality, you will find in textbooks that denitrification involves not
a direct reduction of nitrate into N\textsubscript{2} or
N\textsubscript{2}O, but rather a sequence of reductions, summarized in
equation \eqref{eq:denit-reduc-suite}, where nitrite, nitrogen monoxide,
and nitrous oxides are intermediate products:

\begin{equation}
NO_3^- \rightarrow NO_2^- \rightarrow NO \rightarrow N_2O \rightarrow N_2
\label{eq:denit-reduc-suite}
\end{equation}

Nitrous oxide is thus evidence of an incomplete denitrification. Because
denitrification currently is the one mechanism, which removes nitrogen
from the aqueous phase as gaseous byproduct, it currently ranks as the
most effective ways to treat excess nitrogen in water. Entire research
programs are dedicated to the development of methods and treatment
systems to optimize this process. One of the active research areas is
about finding ways to have denitrification go all the way to the
N\textsubscript{2} stage to minimize the production of
N\textsubscript{2}O.

Because this is a reduction process, and because nitrate is \emph{not
assimilated} in the denitrifier cells, a very short and good definition
of denitrification is the \emph{microbially mediated dissimilatory
reduction of nitrate into dinitrogen}. Nitrate reduction (= gains
electrons) in denitrification is illustrated in the
\protect\hyperlink{redox}{redox half reaction}
\eqref{eq:denit-half-reaction}:

\begin{equation}
2 NO_3^- + 10 e^- + 12 H^+ \rightleftharpoons N_2 + 6 H_2O
\label{eq:denit-half-reaction}
\end{equation}

If you look carefully, you can see that to have the half-reaction
balanced, 10 electrons were added for 2 nitrate molecules, or 5
electrons per nitrate molecule. This should ring a bell to you! Remember
when we allocate electrons on N for dinitrogen and nitrate molecules,
each N for dinitrogen has 5 electrons for itself, and the N of nitrate
has 0 electrons for itself. So nitrogen needs to acquire 5 electrons to
go from nitrate to dinitrogen. And guess what? This is exactly what
half-reaction \eqref{eq:denit-half-reaction} shows! So now, if you were
not convinced of the importance of the electron allocation in the early
chapters, maybe you see its use now.

The overall transfer of electrons from the organic matter to
NO\textsubscript{3}\textsuperscript{-}, can be written as the
combination of equation \eqref{eq:denit-half-reaction} and equation
\eqref{eq:OM-edonor}, to yield, let us admit it a rather complicated
equation:

\begin{equation}
5 CH_2O + 4 NO_3^- + 4 H^+ \rightarrow 5 CO_2 + 2 N_2 + 7 H_2O
\label{eq:denit-reduc}
\end{equation}

A more complete and more descriptive definition of denitrification is

\begin{quote}
Denitrification refers to the dissimilatory reduction, by essentially
aerobic bacteria, of one or both of the ionic nitrogen oxides (nitrate,
NO\textsubscript{3}\textsuperscript{-}, and nitrite,
NO\textsubscript{2}\textsuperscript{-}) to the gaseous oxides (nitric
oxide, NO, and nitrous oxide, N\textsubscript{2}0), which may themselves
be further reduced to dinitrogen (N\textsubscript{2}). The nitrogen
oxides act as terminal electron acceptors in the absence of oxygen. The
gaseous nitrogen species are major products of these reductive
processes. \citep{Knowles1982-ku}
\end{quote}

Denitrification is thought to be inhibited by the presence of oxygen,
and thus only occurs in our theoretical wetland soil, below the aerobic
layer. However, denitrification only occurs if there is nitrate present
as electron acceptor. Therefore, for denitrification to proceed, there
must be a supply of nitrate that can compensate the demand due to
denitrification. The only place from where nitrate can be supplied, is
the water column, and, possibly, the aerobic layer of the sediment where
nitrification can take place, as we shall
\protect\hyperlink{nitrification}{see later}.

Following the same analysis as that of aerobic respiration, the demand
for nitrate creates a \textbf{downward flux} of nitrate from the water
column down. Because of the transport distance for nitrate from the
water column and through the aerobic layer of the sediment, the supply
tends to be lower than the demand. This suggests that at one point in
depth there will be no more nitrate as they are consumed faster than
they can diffuse downward. Nitrates thus diffuse through layer 1, and
both diffuse and are consumed as electron acceptors in layer 2 in Figure
\ref{fig:wetland-soil-layers}. Again, because the demand exceeds the
supply, nitrates cannot move further down than the bottom of layer 2.
Denitrification is thus restricted to layer 2.

\section{Manganese and Iron oxides
reductions}\label{manganese-and-iron-oxides-reductions}

Below the denitrification layer, microbes have to find new electron
acceptors for their respiration. After O\textsubscript{2}, which is a
gas, NO\textsubscript{3}\textsuperscript{-}, which is a dissolved anion,
the next elements which serve as electron acceptors are solids:
Manganese and Iron oxides. They will serve as electron acceptors, only
if soil minerals containing Mn and Fe oxides are present. In our
theoretical wetland soil, the assumption is that these oxides are
present. In layer 3 (Figure \ref{fig:wetland-soil-layers}),
MnO\textsubscript{2} serve as the electron acceptor following the
general redox half-reaction \eqref{eq:MnO2-reduc-half-reaction}, and
respiration scheme illustrated in Figure \ref{fig:MnO2-resp}.

\begin{equation}
MnO_2 + 4 e^- + 4 H^+ \rightleftharpoons Mn^{2+} + 2 H_2O
\label{eq:MnO2-reduc-half-reaction}
\end{equation}

The overall transfer of electrons from the organic matter to
MnO\textsubscript{2}, i.e., the combination of equation
\eqref{eq:MnO2-reduc-half-reaction} and equation \eqref{eq:OM-edonor} can be
written as:

\begin{equation}
CH_2O + MnO_2 \rightarrow Mn^{2+} + CO_2 + H_2O 
\label{eq:MnO2-reduc}
\end{equation}

\begin{figure}

{\centering \includegraphics[width=0.75\linewidth]{pictures/respiration-OM-Mn} 

}

\caption{Respiration scheme for heterotrophic Manganese oxide reduction}\label{fig:MnO2-resp}
\end{figure}

While the electron donor is immobile, the reduction of
MnO\textsubscript{2} (equation \eqref{eq:MnO2-reduc-half-reaction})
produces Mn\textsuperscript{2+} ions, which are dissolved, mobile in
water, and can diffuse following concentration gradients. In our
original hypothesis of the sudden flooding of the wetland soil,
MnO\textsubscript{2} oxides would be reduced from layer 3 down. The
Mn\textsuperscript{2+} ions produced in these layers, and the lack of
Mn\textsuperscript{2+} ions in layers 1 and 2, would create a
concentration gradient, which would generate an \textbf{upward flux} of
Mn\textsuperscript{2+}, this time, from layers 3 to 6 into layers 2 and
1. The fate of Mn\textsuperscript{2+} ions as they reach the aerobic
layer is discussed below.

After the MnO\textsubscript{2} oxides are reduced, the iron oxides and
hydroxide minerals will similarly serve as electron donors following the
general equation \eqref{eq:Fe-reduc-half-reaction}, and respiration scheme
illustrated in Figure \ref{fig:Fe-resp}. The reduction of an iron
hydroxide Fe(OH)\textsubscript{3} has been added to show that in
reality, Fe\textsuperscript{3+} never exists as such, but almost always
as iron oxides or hydroxides. Many forms of
\href{https://en.wikipedia.org/wiki/Iron_oxide}{iron oxides} exist in
soils, hence the choice of choosing iron hydroxide
Fe(OH)\textsubscript{3} in equation \eqref{eq:Fe-reduc-half-reaction}.
Fe\textsuperscript{3+}, Fe(OH)\textsubscript{3}, and other iron oxides
are referred to as \emph{\textbf{ferric iron}} or Fe(III), because their
\protect\hyperlink{oxidation-state}{oxidation state} is 3. Ferric iron
generally has an orange rusty color.

\begin{align}
Fe^{3+} + 1 e^- & \rightleftharpoons & Fe^{2+} \\
Fe(OH)_3 + 1 e^- + 3 H^+ & \rightleftharpoons & Fe^{2+} + 3 H_2O
\label{eq:Fe-reduc-half-reaction}
\end{align}

The overall transfer of electrons from the organic matter to ferric iron
can be written as the combination of equation
\eqref{eq:Fe-reduc-half-reaction} and equation \eqref{eq:OM-edonor} to
yield:

\begin{equation}
CH_2O + 4 Fe^{3+} + H_2O \rightarrow 4 Fe^{2+} + CO_2  + 4 H^+
\label{eq:Fe-reduc}
\end{equation}

\begin{figure}

{\centering \includegraphics[width=0.75\linewidth]{pictures/respiration-OM-Fe} 

}

\caption{Respiration scheme for heterotrophic Iron oxide reduction}\label{fig:Fe-resp}
\end{figure}

Similarly to Mn\textsuperscript{2+} ions, Fe\textsuperscript{2+} ions,
or \emph{\textbf{ferrous iron}}, or Fe(II) is a dissolved iron which is
mobile in water, and can diffuse following concentration gradients. For
the same reasons explained above for Mn\textsuperscript{2+}, an upward
concentration gradient between zones 4 to 6 and layers 3 to 1 is going
to appear and the Fe\textsuperscript{2+} ions will tend to diffuse
\textbf{upward} through layers 3 and 2. The fate of
Fe\textsuperscript{2+} ions as they reach the aerobic layer is discussed
below.

This vertical spatial sequence of layers 3 and 4 presented here probably
only applies not too long after the theoretical wetland soil has been
flooded. Indeed, over the long periods, the supply of ferric iron and
manganese oxides will run out, as the only supply is in immobile mineral
forms. So over long periods, layers 3 and 4 do not exist. Is the case of
a net downward water infiltration, the Fe\textsuperscript{2+} and
Mn\textsuperscript{2+} ions diffuse upward all the way into the aerobic
layer, and accumulate there for reasons illustrated below. In the more
likely case of small but real net downward flux of water, the dissolved
ions will leach out of the soil profile. The consequences are that
poorly drained soils then to leach out their iron and manganese, which
is referred to as \emph{\textbf{iron and manganese depletion}}. This is
the reason for the grey color of hydric soils (Figure
\ref{fig:Fe-depletion}).

\begin{figure}

{\centering \includegraphics[width=0.75\linewidth]{pictures/hydric-soil} 

}

\caption{Example of pale bluish gray redox depletions. Note the faint rusty orange concentration distributed throughout the soil matrix. Reproduced with permission © 2012 Nature Education All rights reserved.}\label{fig:Fe-depletion}
\end{figure}

\section{Sulfate reduction}\label{sulfate-reduction}

After all previous electron acceptors have been used, sulfate becomes
the next electron acceptor. As we have seen before, the sulfur atom on
SO\textsubscript{4}\textsuperscript{2-} has zero electrons for itself
and it therefore can accept electrons following
\href{redox-half-reactions}{redox half-reaction} equation
\eqref{eq:SO4-reduc-half-reaction}.

\begin{equation}
SO_4^{2-} + 8 e^- + 10 H^+ \rightleftharpoons H_2S + 4 H_2O 
\label{eq:SO4-reduc-half-reaction}
\end{equation}

Here again, to balance the half-reaction, 8 electrons were needed. And
by now, this should not surprise you because you now remember that the
electron allocation rules tell you that on sulfate and dihydrogen
sulfide, the S atom respectively has 0 and 8 electrons for itself. So
the reduction of sulfate into dihydrogen sulfide requires the addition
of 8 electrons, and guess what? Half-reaction
\eqref{eq:SO4-reduc-half-reaction} confirms just that.

While the electron donor is still organic matter as shown in Figure
\ref{fig:SO4-resp}, the overall transfer of electron from the OM to
sulfate can be written as:

\begin{equation}
2 CH_2O + SO_4^{2-} + 2 H^+ \rightarrow H_2S + 2 CO_2  + 2 H_2O 
\label{eq:SO4-reduc}
\end{equation}

\begin{figure}

{\centering \includegraphics[width=0.75\linewidth]{pictures/respiration-OM-SO4} 

}

\caption{Respiration scheme for heterotrophic sulfate reduction}\label{fig:SO4-resp}
\end{figure}

Typical concentrations of sulfate in ground- and stream waters are
between 1 and 10 mg SO\textsubscript{4}\textsuperscript{2-}/L. In our
theoretical wetland soil profile, sulfate that might be originally
present in the porewater will be reduced in layers 5 and 6 of Figure
\ref{fig:wetland-soil-layers}. For the same reasons invoked for oxygen
and nitrate, the potential source of supply for sulfate for the sulfate
reducing layer is all the sulfate that might be present in the layers
above and the water column. The demand for sulfate in layer 5 will
create a downward concentration gradient which will generate a downward
diffusive movement of sulfate down to layer 5. And again, because of all
the diffusion distance, the supply of
SO\textsubscript{4}\textsuperscript{2-} is limited and does not match
the demand. The imbalance between the sulfate supply and demand will
limit the diffusion of sulfate to the bottom of layer 5, below which
there will be no more sulfate.

\section{The methanogenesis oddity}\label{the-methanogenesis-oddity}

The last \protect\hyperlink{redox}{redox} reactions to take place at the
bottom of our theoretical wetland soil uses CO\textsubscript{2} as the
electron acceptor. Interestingly, the generic organic matter is no
longer the electron donor, but is replaced by two byproducts of
fermentation processes: H\textsubscript{2} and acetic acid
CH\textsubscript{3}COOH as illustrated in Figure \ref{fig:CO2-resp}.

\begin{figure}

{\centering \includegraphics[width=0.75\linewidth]{pictures/respiration-OM-CO2} 

}

\caption{Respiration scheme for the heterotrophic Carbon dioxide reduction or methanogenesis}\label{fig:CO2-resp}
\end{figure}

The fermentation products are beyond the scope of this class so, we will
not come back to that, but it is important to recognize that
H\textsubscript{2} and CH\textsubscript{3}COOH can donate electrons as
shown in these half-reactions:

\begin{equation}
H_2 \rightleftharpoons 2H^+ + 2 e^- 
\label{eq:H2-half-reaction}
\end{equation}

and

\begin{equation}
CH_3COOH + 2 H_2O \rightleftharpoons 2 CO_2 + 8 H^+ + 8 e^- 
\label{eq:acetic-acid-half-reaction}
\end{equation}

\section{First summary on the electron acceptor chain in wetland
soils}\label{first-summary-on-the-electron-acceptor-chain-in-wetland-soils}

\begin{itemize}
\tightlist
\item
  except for methanogenesis, organic matter always serves, for
  \protect\hyperlink{trophic-names}{organotrophs}, as the electron
  donor. Moreover, it is \emph{\textbf{exclusively}} the Carbon of the
  OM which provides the electrons.
\item
  as a result, the byproducts of the oxidation of the OM when the
  organic carbons lose their electrons are always the same:
  CO\textsubscript{2}, NH\textsubscript{4}\textsuperscript{+},
  H\textsubscript{2}S/HS\textsuperscript{-}, and
  PO\textsubscript{4}\textsuperscript{3-} (but for methanogenesis). This
  means that at one point, all of these four (five if one counts both
  H\textsubscript{2}S and HS\textsuperscript{-}, their exact proportion
  depends on pH, see \protect\hyperlink{hydrogen-sulfide}{details in the
  glossary}) molecules will accumulate where they are produced unless
  they are used by another process
\item
  To the contrary, the byproducts of the reduction of the electron
  acceptors are variable and depend on the electron acceptor. In the
  case of incomplete denitrification, and in the case of methanogenesis,
  N\textsubscript{2}O and CH\textsubscript{4} are significant
  \protect\hyperlink{GHG}{greenhouse gases}.
\end{itemize}

\section{Supply and demand of electron acceptors and of the byproducts
of Organic Matter
oxidation}\label{supply-and-demand-of-electron-acceptors-and-of-the-byproducts-of-organic-matter-oxidation}

Now that we have established the reduction processes of the different
electron acceptors at play, let us look at the consequences of the
demands and the supplies associated with the respiratory processes.

\subsection{Demands drive downward fluxes of dioxygen, nitrate and
sulfate}\label{demands-drive-downward-fluxes-of-dioxygen-nitrate-and-sulfate}

In the Figure \ref{fig:diagenesis-diffusion-directions} below, the
general directions of the fluxes of electrons acceptors and byproducts
of respiration are illustrated for our theoretical wetland soil. The
processes taken together is sometimes referred to as soil or sediment
\textbf{diagenesis} and all the processes in there are then collectively
referred to as \textbf{diagenetic processes}.

\begin{figure}

{\centering \includegraphics[width=1\linewidth]{pictures/diagenesis-diffusion-directions} 

}

\caption{Diffusion fluxes of electron acceptors and all other soil diagenesis processes of a theoretical layered wetland soil}\label{fig:diagenesis-diffusion-directions}
\end{figure}

The demands for dioxygen, nitrate, and sulfate in their respective
layers, lower their concentrations compared to the overlying water and
layers, hence the formation of concentration gradients, which then drive
downward diffusive fluxes of these electron acceptors to their
respective layers. These fluxes are represented as downward arrows in
Figure \ref{fig:diagenesis-diffusion-directions}.

Now, it is the imbalance between the supply from above and the demand
from below, that explain why the downward diffusion does not go beyond
the bottom of the respective layers 1, 2, and 5 in Figure
\ref{fig:diagenesis-diffusion-directions}. The limitation of supply has
been described for dioxygen in section
\ref{an-aerobic-layer-near-the-soil-water-interface}. The diffusivity,
distance, and tortuosity of the soil pores also apply for nitrate and
sulfate, and explain why the demand is generally not met by the supply.

\subsection{Supply of byproducts of organic matter
oxidation}\label{supply-of-byproducts-of-organic-matter-oxidation}

For every respiration process described above (except for
methanogenesis), the byproducts are: CO\textsubscript{2},
NH\textsubscript{4}\textsuperscript{+},
H\textsubscript{2}S/HS\textsuperscript{-}, and
PO\textsubscript{4}\textsuperscript{3-}. This suggests that in every
single layer, there is a supply of all for of these byproducts.
Inevitably, this will create an upward concentration gradient, which
will then be followed by an upward flux of these four molecules in the
sediment.

Phosphate will thus dissolve upward, at least until it reaches the
aerobic layer. There it might encounter soil mineral oxides where it
might bind, as we shall see in future chapters, hence the tips of the
arrows not reaching the soil-water interface. However, it is possible
that, if the aerobic layer is very thin and the mineral oxides are
scarce in this thin layer, phosphates may diffuse all the way into the
water column, hence the dotted arrow.

\hypertarget{nitrification}{\subsection{Nitrification}\label{nitrification}}

Similarly, because of ammonium production in all the layers, but that of
methanogenesis, ammonium will diffuse upward, until it reaches the
aerobic layer. Remember that the nitrogen atom on the ammonium carries 8
electrons for itself, so it potentially carries a lot of energy. And
yes, you guessed right, microbes called \textbf{nitrifiers} take
advantage of these electrons and use \textbf{ammonium as their electron
donors}, and use O\textsubscript{2} as their electron acceptor. Because
the electron donor this time is not an organic molecule, nitrifiers are
called \protect\hyperlink{trophic-names}{lithotrophs}. Nitrification is
represented in Figure \ref{fig:diagenesis-diffusion-directions} as the
horizontal white arrow to the left. The nitrification respiration
schemes is summarized in Figure \ref{fig:respiration-NH4-O2}.

\begin{figure}

{\centering \includegraphics[width=0.75\linewidth]{pictures/respiration-NH4-O2} 

}

\caption{Respiration scheme for nitrification}\label{fig:respiration-NH4-O2}
\end{figure}

Depending on the thickness of the aerobic layer and the availability of
dioxygen, all or only a portion of the ammonium may get nitrified as
ammonium moves upward, hence the dotted arrow going all the way to the
water column. As nitrate is now produced in the aerobic layer of the
soil, then the concentration gradient may sway upward or remain
downward, depending on the nitrate concentration in the water column,
hence the upward nitrate dotted arrow in Figure
\ref{fig:diagenesis-diffusion-directions}. You may also notice that a
dotted white arrow pointing the left has been added to illustrate
nitrification which tends to readily occur in the water column, often
thanks to nitrifiers attached at the soil-water interface.

In reality nitrification involves two stages: the oxidation of ammonium
into nitrite, performed by ammonium oxidizing bacteria, of which two
important genera are \emph{Nitrosomonas} and \emph{Nitrosococcus}, and
then the oxidation of nitrite into nitrate, performed by
\emph{Nitrobacter} and \emph{Nitrospira} bacteria (Equation
\eqref{eq:nitrif}). But as far as we are concerned, both steps tend to
occur almost simultaneously, and nitrite is thermodynamically unstable,
and as a result very little tends to accumulate, either in soil or
sediment.

\begin{equation}
NH_4^+ \rightarrow NO_2^- \rightarrow NO_3^-
\label{eq:nitrif}
\end{equation}

\subsection{Gas bubble formation}\label{gas-bubble-formation}

Among the last two OM oxidation byproducts, CO\textsubscript{2} and
H\textsubscript{2}S are gases. We saw earlier, that in reality the
balance between H\textsubscript{2}S and HS\textsuperscript{-}
\protect\hyperlink{H2S}{depends on the pH}. In rather organic soils,
which most treatment wetland soils are, the pH tends to be rather
acidic, often below 6.5. So it is fair to represent the
H\textsubscript{2}S/HS\textsuperscript{-} couple as H\textsubscript{2}S
rather than HS\textsuperscript{-} (see
\protect\hyperlink{H2S}{H\textsubscript{2}S/HS\textsuperscript{-}
Figure}), hence the choice of H\textsubscript{2}S in Figure
\ref{fig:diagenesis-diffusion-directions} and the use of
H\textsubscript{2}S below.

So in the end, if the summation of partial pressure of these and all
other dissolved gases exceeds 1 atm + the hydraulic head, gas bubble
will form and migrate upward. But because of surface tension forces, gas
bubbles tend to get rather easily trapped in wetland soils and sediment.
Hence the release of wetland gases when somebody or something disturbs
the sediment, as we have seen in lab. To these two gases, one should add
the production of CH\textsubscript{4} in the methanogenesis layer, which
will readily `join' the gas bubbles and the ride with them. The
dissolved fraction will also tend to move upward because of the
concentration gradient. Interestingly, methane does not oxidized very
well in normal aerobic conditions of wetland soils, and will therefore
tend to diffuse all the way up to the water column as illustrated in
Figure \ref{fig:diagenesis-diffusion-directions}.

Because of the demand for CO\textsubscript{2} in the methanogenesis
layer, downward arrows have been added for layers 4 and 5 in Figure
\ref{fig:diagenesis-diffusion-directions}. The downward diffusion would
only apply for the dissolved CO\textsubscript{2} as the gaseous form
would obviously tend to move upward.

\subsection{Oxidation of upward moving reduced
sulfur}\label{oxidation-of-upward-moving-reduced-sulfur}

Similarly to ammonium, and although a good proportion of
H\textsubscript{2}S will end up in the gas phase, a still significant
amount will stay in solution and will diffuse upward until it reaches
the aerobic layer. Very similarly to ammonium, H\textsubscript{2}S still
carries 8 electrons, which can be used for respiration provided that a
strong enough oxidizer be present. In the aerobic layer of the sediment,
H\textsubscript{2}S will be oxidized back into sulfate following the
respiration scheme in Figure \ref{fig:respiration-H2S-O2}

\begin{figure}

{\centering \includegraphics[width=0.75\linewidth]{pictures/respiration-H2S-O2} 

}

\caption{Respiration scheme for hydrogen sulfide oxidation}\label{fig:respiration-H2S-O2}
\end{figure}

The bacteria taking advantage of the electrons on sulfur of
H\textsubscript{2}S are called \textbf{sulfur oxidizing bacteria}. There
are different from the \textbf{sulfur oxidizing bacteria} which use OM
as their electron donors, and sulfate as their electron acceptors. The
oxidation of H\textsubscript{2}S has been summarized by the white
horizontal arrow in both the aerobic layer of the soil and the water
column. Sulfate produced can then diffuse back downward to layer 5 in
Figure \ref{fig:diagenesis-diffusion-directions}.

\subsection{\texorpdfstring{Oxidation of upward moving
Mn\textsuperscript{2+} and
Fe\textsuperscript{2+}}{Oxidation of upward moving Mn2+ and Fe2+}}\label{oxidation-of-upward-moving-mn2-and-fe2}

The last direct reduction byproducts of respiration processes include
Mn\textsuperscript{2+} and Fe\textsuperscript{2+}. Because they are
being produced in layers 3 and below, an upward concentration gradient
will be created, followed by an upward movement of
Mn\textsuperscript{2+} and Fe\textsuperscript{2+}. In Figure
\ref{fig:diagenesis-diffusion-directions}, both of them have been
represented as starting to move up from layers 4. In all logic,
Mn\textsuperscript{2+} and Fe\textsuperscript{2} should start moving
upward as low as layer 6 because our original hypothesis was that soil
was suddenly flooded, so in the temporal sequence of electron acceptors,
before sulfate carbon dioxide reduction conditions would prevail in
Layers 5 and 6, respectively, the Mn and Fe oxides will serve as
electron acceptors, and Mn\textsuperscript{2+} and
Fe\textsuperscript{2+} will then be produced.

The more important fact is the fate of both Mn\textsuperscript{2+} and
Fe\textsuperscript{2+} as they reach the aerobic layer. Because they are
both reduced, they potentially have one electron to give, and yes, some
bacteria have specialized in the ability to oxidize these ions. These
bacteria are referred to as manganese and iron oxidizing bacteria. The
respiration scheme corresponding to iron oxidation is illustrated in
Figure \ref{fig:respiration-Fe-O2} below.

\begin{figure}

{\centering \includegraphics[width=0.75\linewidth]{pictures/respiration-Fe-O2} 

}

\caption{Respiration scheme for iron oxidation}\label{fig:respiration-Fe-O2}
\end{figure}

As both Mn\textsuperscript{2+} and Fe\textsuperscript{2+} get oxidized
in the aerobic layer of the sediment, they form Mn and Fe oxides, which
are solids and precipitates with the other soil minerals, hence the
underlying and downward pointing arrows under the MnO\textsubscript{2}
and Fe\textsuperscript{3+} in Figure
\ref{fig:diagenesis-diffusion-directions}. The precipitation of iron
oxides or hydroxides can be quite visually spectacular sometimes as
reduced groundwater seeps out into the open. Iron can get very quickly
oxidized and form
\href{https://photos.app.goo.gl/NUPx5DjU1QdZC09q1}{rusty biofilms}.

\begin{figure}

{\centering \includegraphics[width=0.75\linewidth]{pictures/Fe-seep-claridge} 

}

\caption{Picture of an 'iron seep' in Goldsboro, NC, as reduced groundwater gets oxidized in contact with air}\label{fig:Fe-seep-CL}
\end{figure}

\subsection{Moving of Dissolved Organic
Carbon}\label{moving-of-dissolved-organic-carbon}

The last but not least of the processes at play in diagenesis is the
formation and diffusion of Dissolved Organic Carbon in poorly or anoxic
soils. This is the subject of a future chapter.

This concludes this long chapter on the importance respiration processes
occurring in wetland soils.

\hypertarget{glossary}{\chapter{Glossary}\label{glossary}}

This glossary is meant to assemble terms that we routinely use in
Environmental Sciences and Engineering and which are expected to be
mastered by students taking BAE 204 at NC State university. They are
ordered in alphabetical list for better retrieval and look up.

\section{A}\label{a}

\subsection{Aerobic respiration}\label{aerobic-respiration}

\subsection{Anaerobic respiration}\label{anaerobic-respiration}

\subsection{Ammonia}\label{ammonia}

\begin{itemize}
\tightlist
\item
  \href{https://en.wikipedia.org/wiki/Ammonia}{Ammonia} is a colorless
  gas with a characteristic pungent smell
\item
  Formula: \(NH_3\)
\item
  Ammonia 3D shape:
  \includegraphics[width=0.25000\textwidth]{pictures/Ammonia-3D-balls-A.png}
\item
  Lewis dot structure:
  \includegraphics[width=0.25000\textwidth]{pictures/ammonia-lewis-structure.jpg}
\item
  Number of electron N has for itself following electronegativity rule:
  eight
\item
  \(NH_3\) can only be an \textbf{electron donor}
\item
  Because N has so many electrons to be potentially donated, ammonia is
  generally unstable in an aerobic environment. As a result, it tends to
  trace quantities in nature
\item
  When dissolved in water, and depending on the pH of the solution,
  ammonia converts to \protect\hyperlink{NH4}{ammonium} following the
  reaction:
\end{itemize}

\[
H_2O + NH_3 \rightleftharpoons OH^{-} + NH_4^{+}
\]

\begin{itemize}
\tightlist
\item
  Production:

  \begin{itemize}
  \item
    Because of its many uses, ammonia is one of the most highly produced
    inorganic chemicals. Dozens of chemical plants worldwide produce
    ammonia. Consuming more than 1\% of all man-made power, ammonia
    production is a significant component of the world energy budget.
  \item
    In 2014, about 88\% of the ammonia produced was used for fertilizing
    agricultural crops
  \item
    Modern ammonia-producing plants generally depend on the
    \protect\hyperlink{haber-bosch}{Haber-Bosch process} which consists
    into reducing dinitrogen into ammonia

    \[
    3\,H_2 + N_2 \to 2\,NH_3
    \]
  \end{itemize}
\item
  Consumption:

  \begin{itemize}
  \item
    Ammonia is directly or indirectly the precursor to most
    nitrogen-containing compounds. Virtually all synthetic nitrogen
    compounds are derived from ammonia.
  \item
  \end{itemize}
\end{itemize}

\emph{\protect\hyperlink{top}{back to top}}

\subsection{Ammonium}\label{ammonium}

\begin{itemize}
\tightlist
\item
  \href{https://en.wikipedia.org/wiki/Ammonium}{Ammonium} is the most
  reduced inorganic nitrogenous cation (positively charged).
\item
  Ammonium 3D shape:
  \includegraphics[width=0.25000\textwidth]{pictures/Ammonium-3D-balls.png}
\item
  Lewis dot structure:
  \includegraphics[width=0.25000\textwidth]{pictures/ammonium_lewis_structure.png}
\item
  Number of electron N has for itself following electronegativity rule:
  eight
\item
  It is formed by the protonation of \protect\hyperlink{NH3}{ammonia}
  following the reaction:
\end{itemize}

\begin{equation}
H_2O + NH_3 \rightleftharpoons OH^{-} + NH_4^{+} \label{eq:NH3}
\end{equation}

\begin{itemize}
\tightlist
\item
  The relative abundance of ammonium vs ammonia depends on the pH of the
  solution. See figure below
\end{itemize}

\includegraphics{Textbook_BAE204_files/figure-latex/NH3-NH4-1.pdf}

\begin{itemize}
\tightlist
\item
  Because in most natural aqueous environments, pH is below 8, ammonium
  tends to be the preponderant form.
\item
  Production:

  \begin{itemize}
  \tightlist
  \item
    In nature, ammonium is a waste product of the mineralization of
    organic molecules
  \item
    It is added as fertilizer on soils as
    \href{https://en.wikipedia.org/wiki/Ammonium_nitrate}{ammonium
    nitrate}
  \end{itemize}
\item
  Health hazard:

  \begin{itemize}
  \tightlist
  \item
    Ammonia vapor has a sharp, irritating, pungent odor that acts as a
    warning of potentially dangerous exposure
  \item
    Exposure to very high concentrations of gaseous ammonia can result
    in lung damage and death
  \end{itemize}
\item
  Drinking water standard:
\end{itemize}

\emph{\protect\hyperlink{top}{back to top}}

\subsection{Ammonium nitrate}\label{ammonium-nitrate}

\begin{itemize}
\item
  \href{https://en.wikipedia.org/wiki/Ammonium_nitrate}{Ammonium
  nitrate} is a chemical compound, the nitrate salt of the ammonium
  cation
\item
  It is a white crystal solid and is highly soluble in water.
\item
  Formula: \(NH_4NO_3\)
\item
  3D shape:
  \includegraphics[width=0.33000\textwidth]{pictures/Ammonium-nitrate-xtal-3D-balls-A.png}
\item
  It is predominantly used in agriculture as a high-nitrogen fertilizer
\item
  Its other major use is as a component of explosive mixtures used in
  mining, quarrying, and civil construction
\item
  Production:

  \begin{itemize}
  \tightlist
  \item
    Ammonium nitrate does exist naturally in mines of the
    \href{https://en.wikipedia.org/wiki/Atacama_Desert}{Atacama desert}
    in Chile but globally nearly all ammonium nitrate is now produced
    synthetically
  \item
    Byproduct of all respiration processes. Ammonium ions are a waste
    product of the metabolism of animals. In fish and aquatic
    invertebrates, it is excreted directly into the water. In mammals,
    sharks, and amphibians, it is converted in the urea cycle to urea,
    because urea is less toxic and can be stored more efficiently. In
    birds, reptiles, and terrestrial snails, metabolic ammonium is
    converted into uric acid, which is solid and can therefore be
    excreted with minimal water loss.
  \end{itemize}
\item
  Consumption/utilization:

  \begin{itemize}
  \tightlist
  \item
    It is used as a fertilizer, because it tends to release inorganic
    nitrogen slowly in soil. Applied as a surface fertilizer, it
    penetrates the soil with rainfall infiltration. Highly soluble, the
    nitrate anion becomes readily available to plant roots, although it
    is susceptible to leaching below the root system into the shallow
    and deep groundwater. The ammonium cation tends to adsorb to soil
    particles and is thus not as susceptible to leaching. Ammonium can
    be directly uptaken by plant roots, which thermodynamically makes
    sense, although because in most soils aerobic conditions are
    preponderant, nitrate tends to be the ion uptaken most often. Soil
    bacteria in the aerobic zone of the soil will oxidize adsorbed
    ammonium into nitrate, which then becomes available for plant roots.
    The whole chain of events slows the release of inorganic nitrogen to
    crops and thus makes for more effective fertilizers.
  \end{itemize}
\item
  Health hazard

  \begin{itemize}
  \tightlist
  \item
    No direct known health hazard
  \end{itemize}
\end{itemize}

\emph{\protect\hyperlink{top}{back to top}}

\subsection{Anoxic waters}\label{anoxic-waters}

\begin{itemize}
\tightlist
\item
  Anoxic waters are areas of sea water, fresh water, or groundwater that
  are depleted of dissolved oxygen and are a more severe condition of
  hypoxia (Wikipedia)
\item
  Anoxic waters result from an \emph{\textbf{IMBALANCE}} between oxygen
  supply and demand
\end{itemize}

\emph{\protect\hyperlink{top}{back to top}}

\section{C}\label{c}

\subsection{Carbon dioxide}\label{carbon-dioxide}

\begin{itemize}
\item
  Carbon dioxide is a colorless gas which density is 50\% higher than
  that of dry air.
\item
  Formula: \(CO_2\)
\item
  Carbon dioxide 3D shape:
  \includegraphics[width=0.25000\textwidth]{pictures/Carbon_dioxide_3D_ball.png}
\item
  Lewis dot structure:
  \includegraphics{pictures/CO2_lewis_structure.png}
\item
  Number of electron C has for itself following electronegativity rule:
  zero
\item
  \(CO_2\) can only be an electron acceptor
\item
  Production:

  \begin{itemize}
  \tightlist
  \item
    oxidation of C in all organic molecules
  \item
    Almost all respiratory processes on earth (some respiration does not
    involve oxidation of C)
  \item
    Combustion of all Carbon-based fuel
  \end{itemize}
\item
  Consumption:

  \begin{itemize}
  \tightlist
  \item
    Photosynthesis

    \begin{itemize}
    \tightlist
    \item
      Atmospheric carbon dioxide \textbf{is the primary carbon source
      for life} on Earth
    \end{itemize}
  \item
    Calcite precipitation in the oceans
  \end{itemize}
\item
  Ecological Significance:

  \begin{itemize}
  \tightlist
  \item
    Greenhouse Gas, which serves as reference for all other GHG
  \item
    Concentration in the atmosphere \textasciitilde{}380 ppm on the
    rise, or 0.38\%, or a partial pressure of 0.38 atm
  \end{itemize}
\end{itemize}

\begin{figure}
\centering
\includegraphics{pictures/CO2_atm_concentrations.png}
\caption{Carbon Dioxide variations through ancient and modern times}
\end{figure}

\href{https://https://www.epa.gov/sites/production/files/2016-08/documents/print_ghg-concentrations-2016.pdf}{concentrations
of carbon dioxide in the atmosphere from hundreds of thousands of years
ago through 2015, measured in parts per million (ppm). The data come
from a variety of historical ice core studies and recent air monitoring
sites around the world. Each line represents a different data source}
\citep{Epa2016-yj}

\begin{itemize}
\item
  In reality, concentrations are not stable, and vary widely in time and
  in space at the next two videos nicely show.
\item
  The next one results from the model simulations
\end{itemize}

\href{https://www.youtube.com/embed/WGHkY0E4FMY}{Youtube video of
\(CO_2\) modeled seasonal variations}

 - The following one is the combination of both models and observations

\href{https://www.youtube.com/embed/2BWWrJr6TJw}{Youtube video of
\(CO_2\) modeled and observed seasonal variations}

\emph{\protect\hyperlink{top}{back to top}}

\subsection{Carbonates}\label{carbonates}

\begin{itemize}
\tightlist
\item
  After carbon dioxide dissolves in water, it will combine with water to
  form carbonic acid (\(H_2CO_3\)).
\item
  Carbonate serves as \textbf{the carbon source} for aquatic vegetation
\item
  Carbonic acid can then dissociate into bicarbonate (\(HCO_3^-\)) and
  carbonate (\(CO_3^{2-}\))
\end{itemize}

\begin{align}
H_2CO_3  & \rightleftharpoons & HCO_3^- + H^+  \label{eq:H2CO3} \\
HCO_3^- & \rightleftharpoons & CO_3^{2-} + H^+ \label{eq:HCO3}
\end{align}

\begin{itemize}
\tightlist
\item
  In an environment not open to the atmosphere (or where direct exchange
  with the atmosphere is very limited like in stream or wetland
  sediment), the preponderant form depends on the pH and can be
  calculated as illustrated on the graph below.
\end{itemize}

\includegraphics{Textbook_BAE204_files/figure-latex/CO3-1.pdf}

\begin{itemize}
\tightlist
\item
  In sea water, Carbonate can combine with \(Ca^{2+}\) to form Calcium
  Carbonate (\(CaCO_3\)), which precipitates out of solution. In other
  words, calcium carbonate formation is a sink for carbonate, and
  ultimately from \(CO_2\) addition from the atmosphere to to increased
  \(CO_2\) concentrations in the atmosphere.
\item
  Carbonates are thus a great pH buffer in aquatic environments
\end{itemize}

\hypertarget{catabolism}{\subsection{Catabolism}\label{catabolism}}

\begin{itemize}
\tightlist
\item
  Reactions involving the breaking down of organic substrates, typically
  by oxidative breakdown, to provide chemically available energy (e.g.~A
  TP) and/or to generate metabolic intermediates used in subsequent
  anabolic reactions \citep{De_Bolster1997-ul}.
\item
  Synonyms:

  \begin{itemize}
  \tightlist
  \item
    Aerobic and anaerobic \emph{respirations} which use organic
    molecules as electron donors, are synonyms of catabolism.
  \item
    In soil science another synonym of catabolism is
    \emph{mineralization}, which refers to the decomposition or
    oxidation of the chemical compounds in organic matter releasing the
    nutrients contained in those compounds into soluble inorganic forms
    that may be plant-accessible \citep{Wikipedia_contributors2018-ew}.
  \end{itemize}
\end{itemize}

\section{D}\label{d}

\hypertarget{denitrification}{\subsection{Denitrification}\label{denitrification}}

\begin{itemize}
\tightlist
\item
  The \emph{microbially mediated dissimilatory reduction of nitrate into
  dinitrogen}
\item
  see \protect\hyperlink{denit}{denitrification in chapters} for more
  details
\end{itemize}

\subsection{Dihydrogen sulfide}\label{dihydrogen-sulfide}

\begin{itemize}
\tightlist
\item
  see \protect\hyperlink{H2S}{hydrogen sulfide}
\end{itemize}

\emph{\protect\hyperlink{top}{back to top}}

\section{E}\label{e}

\hypertarget{eutrophication}{\subsection{Eutrophication}\label{eutrophication}}

\begin{itemize}
\tightlist
\item
  Definitions:

  \begin{itemize}
  \tightlist
  \item
    `an increase in the rate of supply of organic matter to an
    ecosystem' \citep{Nixon1995-th}
  \item
    `is the enrichment of a water body with nutrients, usually with an
    excess amount of nutrients' (Wikipedia)
  \item
    `the enrichment of water by nutrients, especially nitrogen and/or
    phosphorus, causing an accelerated growth of algae and higher forms
    of plant life to produce an undesirable disturbance to the balance
    of organisms present in the water and to the quality of water
    concerned' \citep{Anonymous1991-ho}
  \item
    `the enrichment of water by nitrogen compounds causing an
    accelerated growth of algae and higher forms of plant life to
    produce an undesirable disturbance to the balance of organisms
    present in the water and to the quality of water concerned'
    \citep{Anonymous1991-xb}
  \end{itemize}
\end{itemize}

\emph{\protect\hyperlink{top}{back to top}}

\section{G}\label{g}

\subsection{\texorpdfstring{Greenhouse gases
\emph{(GHG)}}{Greenhouse gases (GHG)}}\label{greenhouse-gases-ghg}

A greenhouse gas is a gas in an atmosphere that absorbs and emits
radiant energy within the thermal infrared range. This process is the
fundamental cause of the greenhouse effect. The primary greenhouse gases
in Earth's atmosphere are \textbf{water vapor},
\textbf{\protect\hyperlink{CO2}{carbon dioxide}}, \textbf{methane},
\textbf{\protect\hyperlink{N2O}{nitrous oxide}}, and \textbf{ozone}.
Without greenhouse gases, the average temperature of Earth's surface
would be about −18 °C (0 °F), rather than the present average of 15 °C
(59 °F). In the Solar System, the atmospheres of Venus, Mars and Titan
also contain gases that cause a greenhouse effect.
(\href{https://en.wikipedia.org/wiki/Greenhouse_gas}{Wikipedia})

\emph{\protect\hyperlink{top}{back to top}}

\section{H}\label{h}

\subsection{Haber-Bosch process}\label{haber-bosch-process}

\begin{itemize}
\tightlist
\item
  \href{https://en.wikipedia.org/wiki/Haber_process}{Haber-Bosch
  process}
\end{itemize}

\emph{\protect\hyperlink{top}{back to top}}

\hypertarget{H2S}{\subsection{Hydrogen Sulfide}\label{H2S}}

\begin{itemize}
\tightlist
\item
  It is a colorless gas with the characteristic foul odor of rotten
  eggs.
\item
  It is very poisonous, corrosive, flammable and acidic in nature.
\item
  Formula: \(H_2S\)
\item
  hydrogen sulfide \#D shape:
  \includegraphics[width=0.25000\textwidth]{pictures/Hydrogen-sulfide-3D-balls.png}
\item
  Lewis dot structure:
  \includegraphics[width=0.25000\textwidth]{pictures/H2S_Lewis_structure.png}
\item
  Number of electron S has for itself following electronegativity rule:
  eight
\item
  \(H_2S\) can only be an electron donor
\item
  Unstable under aerobic conditions, will readily be oxidized into
  \protect\hyperlink{SO4}{sulfate}
\item
  \(H_2S\) is a polyprotic acid which can lose up to 2 protons in water,
  depending on the pH.
\end{itemize}

\begin{align}
H_2S & \rightleftharpoons & HS^- + H^+ \label{eq:H2S} \\
HS^- & \rightleftharpoons & S^{2-}+ H^+ \label{eq:HS} 
\end{align}

\begin{itemize}
\tightlist
\item
  The figure below suggests that at pH found in most streams (4.5 to 8),
  \(H_2S\) is either preponderant or corresponds to at least 20\% of all
  sulfide forms. \(H_2S\) being a highly volatile product, it explains
  why we can easily smell and detect it in most conditions in streams.
\end{itemize}

\includegraphics{Textbook_BAE204_files/figure-latex/H2S-1.pdf}

\begin{itemize}
\tightlist
\item
  Production:

  \begin{itemize}
  \tightlist
  \item
    Hydrogen sulfide often results from the microbial breakdown, or
    mineralization, of organic matter in anaerobic conditions, such as
    may exist in swamps and sewers. When happening in sediment, this is
    referred to as sediment diagenesis
  \end{itemize}
\item
  Consumption:
\end{itemize}

\emph{\protect\hyperlink{top}{back to top}}

\section{L}\label{l}

\hypertarget{lithotrophs}{\subsection{Lithotrophs}\label{lithotrophs}}

\begin{itemize}
\tightlist
\item
  Lithotrophs are a diverse group of organisms using inorganic substrate
  (usually of mineral origin) to obtain reducing equivalents for use in
  biosynthesis (e.g., carbon dioxide fixation) or energy conservation
  (i.e., ATP production) via aerobic or anaerobic respiration. Known
  chemolithotrophs are exclusively microorganisms.
  \citep{Wikipedia_contributors2018-na}
\end{itemize}

\emph{\protect\hyperlink{top}{back to top}}

\subsection{Limiting factor}\label{limiting-factor}

\emph{\protect\hyperlink{top}{back to top}}

\section{M}\label{m}

\subsection{Methane}\label{methane}

\begin{itemize}
\tightlist
\item
  Under
  \href{https://en.wikipedia.org/wiki/Standard_conditions_for_temperature_and_pressure}{normal
  conditions for temperature and pressure}, methane is a colorless,
  odorless gas main constituent of natural gas, and the simplest alkane
\item
  Formula: \(CH_4\)
\item
  Methane 3D shape:
  \includegraphics[width=0.25000\textwidth]{pictures/Methane-CRC-MW-3D-balls.png}
\item
  Lewis dot structure of methane:
  \includegraphics{pictures/methane_lewis_structure.png}
\item
  Number of electron C has for itself following electronegativity rule:
  eight
\item
  \(CH_4\) can only be an electron \textbf{donor}
\item
  Because methane has so many electrons to give, it will easily `burn'
  in normal atmosphere (provided that ignition T°C be reached, e.g., by
  a spark), liberating large quantities of heat (55.5 MJ/kg). The
  electrons are transferred from the carbon to the oxygen atoms
  following two redox half-reactions to obtain the overall reaction:
\end{itemize}

\begin{align}
CH_4 + 2\,H_2O & \rightleftharpoons & CO_2 + 8\,H^+ + 8\,e^-\\
2\,O_2 + 8\,H^+ + 8\,e^- & \rightleftharpoons & 4\,H_2O\\
& &\\
\hline\\
CH_4 + 2\,O_2 & \to & CO_2 + 2\,H_2O
\end{align}

\begin{itemize}
\item
  Production:
\item
  Consumption:
\item
  Ecological significance:
\end{itemize}

\begin{figure}
\centering
\includegraphics{pictures/CH4_concentrations.png}
\caption{}
\end{figure}

\href{https://www.epa.gov/sites/production/files/2016-08/documents/print_ghg-concentrations-2016.pdf}{Concentrations
of methane in the atmosphere from hundreds of thousands of years ago
through 2015, measured in parts per billion (ppb). The data come from a
variety of historical ice core studies and recent air monitoring sites
around the world. Each line represents a different data
source}\citep{Epa2016-yj}

\begin{itemize}
\tightlist
\item
  Health effects:
\end{itemize}

\subsection{Mineralization}\label{mineralization}

\begin{itemize}
\tightlist
\item
  see \protect\hyperlink{catabolism}{catabolism}
\end{itemize}

\section{N}\label{n}

\subsection{Nitrate}\label{nitrate}

\begin{itemize}
\item
  Nitrate is the stable inorganic nitrogenous anion in oxidized water
\item
  Formula: \(NO_3^{-}\)
\item
  Nitrate 3D shape:
  \includegraphics[width=0.25000\textwidth]{pictures/Nitrate-3D-balls.png}
\item
  Lewis dot structure of nitrate:
  \includegraphics{pictures/nitrate_lewis_structure.png}
\item
  Number of electron N has for itself following electronegativity rule:
  zero
\item
  \(NO_3^{-}\) can only be an electron acceptor
\item
  \(NO_3^{-}\) is technically the conjugate base of nitric acid
  \(HNO_3\), but the \(pk_A\) of the reaction is at a theoretical pH of
  -1.38. In other words, for the pH of most natural waters (4.5
  \textless{} pH \textless{} 8), \(HNO_3\) is totally insignificant.
\item
  Production:

  \begin{itemize}
  \tightlist
  \item
    from the complete \protect\hyperlink{oxidation}{oxidation} of
    inorganic nitrogenous molecules which include
    \protect\hyperlink{NH3}{ammonia}, \protect\hyperlink{NH4}{ammonium},
    \protect\hyperlink{NO2}{nitrite}
  \item
    from the mineralization and complete oxidation of
    \protect\hyperlink{amine}{amine radicals} in organic molecules
  \end{itemize}
\item
  Consumption:

  \begin{itemize}
  \tightlist
  \item
    \textbf{Uptake} from microbes, plants, and algae for their
    \protect\hyperlink{anabolism}{anabolism}, which consists in building
    complex organic molecules from inorganic ones.

    \begin{itemize}
    \tightlist
    \item
      Uptake, assimilation, anabolism, immobilization are all synonymous
      terms to express the fact that the N atom is immobilized, at least
      temporarily in organic molecules.
    \item
      Because N is assimilated in organic molecules during
      uptake/anabolism, and because N gains electrons in the process (it
      is thus \textbf{\emph{reduced}}), we refer to nitrate uptake as
      \textbf{\protect\hyperlink{ANR}{assimilatory nitrate reduction}}.
    \end{itemize}
  \item
    \textbf{\protect\hyperlink{denitrification}{Denitrification}}: under
    anaerobic conditions, nitrate is used by facultative anaerobes as
    electron acceptor to generate \protect\hyperlink{ATP}{ATP} in their
    respiration chain. The two major end products of denitrification are
    gases, namely \protect\hyperlink{N2}{dinitrogen (\(N_2\))} and
    \protect\hyperlink{N2O}{nitrous oxide (\(N_2O\))}, which leave the
    aqueous environment. As such, nitrate is not assimilated by any
    bacteria and denitrification is therefore, as opposed to uptake,
    referred to as **\protect\hyperlink{DNR}{dissimilatory nitrate
    reduction} into \protect\hyperlink{N2}{dinitrogen (\(N_2\))} and
    \protect\hyperlink{N2O}{nitrous oxide (\(N_2O\))}.
  \end{itemize}
\item
  Ecological significance:

  \begin{itemize}
  \tightlist
  \item
    Because of assimilation and denitrification processes, the overall
    nitrate concentrations in rivers tends to diminish from the
    catchment headwaters to the receiving bodies such as estuaries and
    coastal areas. As a result, inorganic nitrogen has naturally been in
    very short supply in these coastal water bodies, and nitrate and
    traditionally been the nutrient limiting aquatic productivity there.
    Very much like with phosphate, algae have adapted to be able to grow
    in very low concentrations. Anthropogenic activities, and
    agriculture in particular, have largely increased the loads and
    concentration of nitrate reaching estuaries, to the point where
    nitrate is no longer the limiting factor. As a result, excess
    nitrate is one of the reasons for the global and persistent presence
    of algal blooms in estuaries and coastal waters.
  \end{itemize}
\item
  Health hazard:

  \begin{itemize}
  \tightlist
  \item
    There is a heated debate about the health hazard that nitrate might
    pose. Some argue that if anything, there might be beneficial
    effects, while others argue that there are evidence of cancers
    linked to excess nitrate absorption. Unfortunately, arguments on
    both sides might not be totally independent of militantism and
    lobbies.
  \item
    The only consensus everybody seems to agree upon is the Blue Baby
    syndrome, or methemoglobinemia. Methemoglobinemia is an unusual and
    potentially fatal condition in which hemoglobin is oxidized to
    methemoglobin and loses its ability to bind and transport oxygen,
    hence the cyanosis (blue appearance) usually visible on fingers,
    toes, and lips. Nitrate reduced to nitrite in the body of humans and
    animals enters the body stream where it seems to directly oxidize
    oxyhemoglobin to methemoglobin-peroxide complex.
  \end{itemize}
\end{itemize}

\href{https://syndromespedia.com/blue-baby-syndrome.html}{\includegraphics[width=0.50000\textwidth]{pictures/BLUE-BABY-SYNDROME.jpg}}

\href{https://syndromespedia.com/blue-baby-syndrome.html}{\textbf{Picture
of a Blue Baby from syndromespedia.com/blue-baby-syndrome.html}}

\begin{itemize}
\tightlist
\item
  Drinking water standards

  \begin{itemize}
  \tightlist
  \item
    Although there is still a heated debate whether or not nitrate does
    have have detrimental health effects, the World Health Organization
    has provided maximum concentration guidelines of 50 mg/L as nitrate
    \citep{World_Health_Organization2011-hd}. These guidelines have been
    enacted in hard laws in the US and in Europe. The 50 mg/L as nitrate
    equates 11.2 mg \(NO_3\)-N/L and in the US, the drinking water
    standard is \textbf{10 mg \(NO_3\)-N/L}.
  \end{itemize}
\end{itemize}

\emph{\protect\hyperlink{top}{back to top}}

\hypertarget{nitrous-oxide}{\subsection{Nitrous
Oxide}\label{nitrous-oxide}}

\begin{itemize}
\item
  Commonly known as laughing gas
\item
  Nitrous oxide has significant medical uses, especially in surgery and
  dentistry, for its anesthetic and pain reducing effects.
\item
  Formula: \(N_2O\)
\item
  Nitrous Oxide 3D shape:
  \includegraphics[width=0.25000\textwidth]{pictures/Nitrous-oxide-dimensions-3D-balls.png}
\item
  \includegraphics[width=0.25000\textwidth]{pictures/N2O_lewis_structure.png}
\item
  Number of electron N has for itself following electronegativity rule:

  \begin{itemize}
  \tightlist
  \item
    the first one on the left has 5
  \item
    the middle N has 3
  \end{itemize}
\item
  \(N_2O\) can be both an electron acceptor and an electron donor
\item
  Production:

  \begin{itemize}
  \tightlist
  \item
    \(N_2O\) is produced due to bacterial processes (over 90\%) and
    anthropogenic processes such as burning of fossil fuel

    \begin{itemize}
    \tightlist
    \item
      The two main bacterial processes are \emph{nitrification} and
      \emph{denitrification}
    \item
      Accounting that human activities have enhanced both nitrification
      and denitrification processes, it is estimated that overall, about
      2/3rd of \(N_2O\) production is natural, and about 1/3rd is human
      enhanced
    \end{itemize}
  \end{itemize}
\item
  Consumption:

  \begin{itemize}
  \tightlist
  \item
    Because of all the electrons stored on the two N atoms (5 + 3 = 8),
    nitrous oxide is a potential electron donor and bacteria can use it
    for their respiration processes
  \end{itemize}
\item
  Ecological significance:
\item
  Powerful Greenhouse Gas, 298 times that of \(CO_2\)
  (\href{https://www.epa.gov/sites/production/files/2015-07/documents/emission-factors_2014.pdf}{EPA})
\item
  Concentration in the atmosphere \textasciitilde{}0.0003 ppm or
  \textasciitilde{}325 ppb on the rise
\end{itemize}

\begin{figure}
\centering
\includegraphics{pictures/N2O_concentrations.png}
\caption{}
\end{figure}

\href{https://www.epa.gov/sites/production/files/2016-08/documents/print_ghg-concentrations-2016.pdf}{Concentrations
of nitrous oxide in the atmosphere from hundreds of thousands of years
ago through 2015, measured in parts per billion (ppb). The data come
from a variety of historical ice core studies and recent air monitoring
sites around the world. Each line represents a different data
source})\citep{Epa2016-yj}

\emph{\protect\hyperlink{top}{back to top}}

\section{O}\label{o}

\subsection{Oligotrophication}\label{oligotrophication}

\begin{itemize}
\tightlist
\item
  `a decrease in the rate of supply of organic matter to an ecosystem'
  \citep{Nixon2009-ft}
\end{itemize}

\emph{\protect\hyperlink{top}{back to top}}

\hypertarget{oxidation}{\subsection{Oxidation}\label{oxidation}}

\begin{itemize}
\tightlist
\item
  Oxidation is the \emph{\textbf{loss of electrons}} by a molecule,
  atom, or ion.
\item
  The term oxidation was first used by Antoine Lavoisier to signify
  reaction of a substance with oxygen. Much later, it was realized that
  the substance, upon being oxidized, loses electrons, and the meaning
  was extended to include other reactions in which electrons are lost.
\end{itemize}

\emph{\protect\hyperlink{top}{back to top}}

\hypertarget{oxidation-state}{\subsection{Oxidation
state}\label{oxidation-state}}

\begin{itemize}
\tightlist
\item
  The oxidation state (\emph{OS}), or sometimes referred to as the
  oxidation number, quantifies the number of electrons that an atom has
  gained (expressed as \emph{\textbf{negative}} charge value) or lost
  (expressed as \emph{\textbf{positive}} charge value) compared to the
  number of valence electrons it has in its free form. As a result, it
  can be zero, positive or negative.
\item
  For metal ions, the OS corresponds to its charge
\item
  The \emph{change} in oxidation state is a very powerful tool to
  understand the \protect\hyperlink{redox}{redox} processes involved in
  the environement. But OS is less than ideal when comparing the
  absolute number of electrons one element \emph{has for itself}
\item
  For example, the nitrogen atom N, has 5 valence electrons. In the
  dinitrogen molecule N\textsubscript{2}, the Lewis dot structure
  suggests that each atom share 3 electrons with the other but
  essentially have 5 electrons for themselves as in Figure
  \ref{fig:ElecAlloc-N2}
\end{itemize}

\begin{figure}

{\centering \includegraphics[width=0.2\linewidth]{pictures/ElecAlloc_N2} 

}

\caption{Electron allocation on each of the N atom for the dinitrogen molecule N~2~}\label{fig:ElecAlloc-N2}
\end{figure}

As a result, the number of electron on each atom equals the number of
valence electrons on the free form of N, hence OS = 0. Now in the
nitrous oxide case in Figure \ref{fig:ElecAlloc-N2O}, the nitrogen atom
to the left OS = 0, but for the middle Nitrogen atom, OS = -2, as two
electrons were stripped by oxygen.

\begin{figure}

{\centering \includegraphics[width=0.2\linewidth]{pictures/ElecAlloc_N2O} 

}

\caption{Electron allocation on each of the N atom for the N~2~O molecule}\label{fig:ElecAlloc-N2O}
\end{figure}

The problem with the oxidation state indicator is that it is a relative
number, and not absolute. The electron allocation indicator that
quantifies the number of electrons each atom has for itself is more
abolute and thus comparable. For example, the inorganic molecule with
one C atom stable in an oxidized environment is CO\textsubscript{2}, the
inorganic molecule with one N atom stable in an oxidized environment is
\(NO_3^-\), and the inorganic molecule with one S atom stable in an
oxidized environment is \(SO_4^{2-}\). All three atoms have zero
electrons for themselves as oxygen has `stolen' them (Figure
\ref{fig:ElecAlloc-CO2-NO3-SO4}). So they are all potential electron
acceptors that can accept up to 8 electrons. In a way they are very
similar.

\begin{figure}

{\centering \includegraphics[width=0.2\linewidth]{pictures/ElecAlloc_CO2} \includegraphics[width=0.2\linewidth]{pictures/ElecAlloc_NO3-} \includegraphics[width=0.2\linewidth]{pictures/ElecAlloc_SO42-} 

}

\caption{Electron allocation on each of the C, N, and S atom for the CO~2~, nitrate and sulfate molecules}\label{fig:ElecAlloc-CO2-NO3-SO4}
\end{figure}

But if one calculates the oxidation states for each of the C, N, and S
atoms, for these three molecules, one would find OS\textsubscript{C} =
-4, OS\textsubscript{N} = -5, and OS\textsubscript{S} = -6\ldots{} We
believe this can be very confusing and prefer to use the electron
allocation concept as a more absolute indicator.

\emph{\protect\hyperlink{top}{back to top}}

\section{P}\label{p}

\subsection{Phosphate}\label{phosphate}

\begin{itemize}
\tightlist
\item
  \href{https://en.wikipedia.org/wiki/Phosphate}{Phosphate} is an
  inorganic chemical and a salt-forming anion of phosphoric acid
\item
  Formula: \(PO_4^{3-}\)
\item
  \includegraphics{pictures/phosphate_lewis_structure.jpg}
\item
  Phosphate is one of the anions of the polyprotic acid (i.e., which can
  liberate several protons \(H^{+}\))
\item
  The conjugate bases of phosphate are:
\end{itemize}

\begin{figure}
\centering
\includegraphics[width=1.00000\textwidth]{pictures/phosphate_conjugate_bases.png}
\caption{}
\end{figure}

\begin{itemize}
\tightlist
\item
  All conjugate bases are related through the set acid-base chemical
  equilibria:
\end{itemize}

\begin{align}
H_3PO_4  & \rightleftharpoons & H_2PO_4^- + H^+ \tag{$pK_A$ = 2.12}\\
H_2PO_4^- & \rightleftharpoons & HPO_4^{2-} + H^+ \tag{$pK_A$ = 7.21}\\
HPO_4^{2-}& \rightleftharpoons & PO_4^{3-} + H^+ \tag{$pK_A$ = 12.67}
\end{align}

\begin{itemize}
\tightlist
\item
  The preponderant form of phosphate in a solution also depends on the
  pH following this relationship:
\end{itemize}

\includegraphics{Textbook_BAE204_files/figure-latex/phosphates-1.pdf}

\begin{itemize}
\tightlist
\item
  Production:

  \begin{itemize}
  \tightlist
  \item
    Phosphorus in general and phosphate in practice has remained one of
    the nutrients limiting the most plant productivity on our planet
  \item
    Phosphates are the naturally occurring form of the element
    phosphorus, found in many phosphate minerals
  \item
    Phosphate minerals are mined to obtain phosphorus for use in
    agriculture and industry
  \item
    The largest global producer and exporter of phosphates is Morocco.
  \item
    Within North America, the largest deposits lie in the Bone Valley
    region of central Florida, the Soda Springs region of southeastern
    Idaho, and the coast of North Carolina (near Aurora).
  \end{itemize}
\item
  Consumption:

  \begin{itemize}
  \tightlist
  \item
    Uptake from all primary producer including plants and algae
  \item
    Phosphate can also be immobilized by bacteria
  \item
    In food industry, phosphates help baked goods rise, they act as
    emulsifiers in processed cheese and canned soup, they add flavor to
    cola and color to frozen french fries. They also can be added to
    meat, poultry and seafood to help the protein bind more water,
    making it juicier after freezing and reheating.
  \end{itemize}
\item
  Ecological significance:

  \begin{itemize}
  \tightlist
  \item
    Phosphorus as phosphate naturally is the most limiting factor for
    primary productivity for land and aquatic plants. Because it tends
    to bind to particles, phosphates have accumulated with sediment
    particularly in coastal areas, where phosphate can become available
    again to algae through
    \protect\hyperlink{sediment_diagenesis}{sediment diagenetic
    processes}. As a result, phosphate is generally not considered the
    most limiting factor for algae in estuaries and coastal environment.
    However, it does tend to be the limiting nutrient in freshwater
    receiving bodies such as lakes and reservoirs.
  \item
    Excess phosphate in freshwater receiving bodies has been shown to be
    the nutrient causing some major
    \protect\hyperlink{eutrophication}{eutrophication} problems
    throughout the planet
  \end{itemize}
\end{itemize}

\emph{\protect\hyperlink{top}{back to top}}

\section{R}\label{r}

\subsection{Reactive nitrogen}\label{reactive-nitrogen}

\begin{itemize}
\item
\end{itemize}

\emph{\protect\hyperlink{top}{back to top}}

\hypertarget{redox}{\subsection{Redox}\label{redox}}

\begin{itemize}
\tightlist
\item
  Redox (short for reduction--oxidation reaction) is a chemical reaction
  in which the oxidation states of atoms are changed. Any such reaction
  involves both a \protect\hyperlink{reduction}{reduction} process and a
  complementary \protect\hyperlink{oxidation}{oxidation} process,
  effectively allowing the electron transfer processes.
\end{itemize}

\subsection{Redox couple}\label{redox-couple}

\begin{itemize}
\tightlist
\item
  redox couples, commonly noted as `Ox/Red' refer to the two forms that
  an element might take depending on its
  \protect\hyperlink{oxidation-state}{oxidation state}
\item
  Ox in `Ox/Red' refers to as the \emph{oxidizer} or oxidizing agent, as
  it has the capability to make other elements \textbf{lose} their
  electrons.
\item
  Red in `Ox/Red' refers to as the \emph{reducer} or reducing agent, as
  it has the capability to \textbf{give} electrons to other elements
\item
  Common redox couples which matter in environmental and ecological
  engineering include:

  \begin{itemize}
  \tightlist
  \item
    CO\textsubscript{2}/CH\textsubscript{2}O
  \item
    O\textsubscript{2}/H\textsubscript{2}O
  \item
    NO\textsubscript{3}\textsuperscript{-}/N\textsubscript{2}
  \item
    NO\textsubscript{3}\textsuperscript{-}/NH\textsubscript{4}\textsuperscript{+}
  \item
    NO\textsubscript{3}\textsuperscript{-}/NO\textsubscript{2}\textsuperscript{-}
  \item
    MnO\textsubscript{2}/Mn\textsuperscript{2+}
  \item
    Fe\textsuperscript{3+}/Fe\textsuperscript{2+}
  \item
    SO\textsubscript{4}\textsuperscript{2-}/H\textsubscript{2}S
  \item
    CO\textsubscript{2}/CH\textsubscript{4}
  \end{itemize}
\end{itemize}

\subsection{Redox half-reactions}\label{redox-half-reactions}

\begin{itemize}
\tightlist
\item
  A half reaction is either the oxidation or reduction reaction
  component of a redox reaction. A half reaction is obtained by
  considering the change in oxidation states of individual substances
  involved in the redox reaction \citep{Wikipedia_contributors2018-be}
\item
  Redox half reactions illustrate the transfer of electrons from the
  oxidized form of an element to a reduced form of an element.
\item
  Half-reactions are often used as a method of balancing redox reactions
\item
  To derive half-reactions,

  \begin{itemize}
  \tightlist
  \item
    the first convention is that the oxidizer is on the left of the
    equation, and the reducer on the right
  \item
    one must first equilibrate all elements other than O and H
  \item
    then one balances for O, by adding H\textsubscript{2}O on the
    appropriate side of the half reaction
  \item
    then one adds the appropriate number of H\textsuperscript{+} to
    balance the H
  \item
    Finally one adds the appropriate number of e\textsuperscript{-} to
    balance the charge. The number of e\textsuperscript{-} must
    correspond to the change in
    \protect\hyperlink{oxidation-state}{oxidation state}
  \end{itemize}
\item
  The half-reactions corresponding to the redox couples above are:
\item
  CO\textsubscript{2}/CH\textsubscript{2}O
\end{itemize}

\begin{equation}
CO_2 + 4 H^+ + 4 e^- \rightleftharpoons CH_2O + H_2O  
\end{equation}

\begin{itemize}
\tightlist
\item
  CH\textsubscript{2}O here is a generic formula to represent the
  electron donating capacity of OM. This equation might be misleading as
  it corresponds to the OM donating 4 electrons in the respiration
  schemes, but because it is written using the ox/red convention, is the
  opposite of the intuitive direction. So a more intuitive half-reaction
  might be, and is effectively what is used to write the full redox
  reaction:
\end{itemize}

\begin{equation}
CH_2O + H_2O \rightleftharpoons CO_2 + 4 H^+ + 4 e^-
\end{equation}

\begin{itemize}
\tightlist
\item
  O\textsubscript{2}/H\textsubscript{2}O
\end{itemize}

\begin{equation}
O_2 + 4 H^+ + 4 e^- \rightleftharpoons 4H_2O  
\end{equation}

\begin{itemize}
\tightlist
\item
  NO\textsubscript{3}\textsuperscript{-}/N\textsubscript{2}
\end{itemize}

\begin{equation}
2 NO_3^- + 10 e^- + 12 H^+ \rightleftharpoons N_2 + 6 H_2O
\end{equation}

\begin{itemize}
\tightlist
\item
  MnO\textsubscript{2}/Mn\textsuperscript{2+}
\end{itemize}

\begin{equation}
MnO_2 + 4 e^- + 4 H^+ \rightleftharpoons Mn^{2+} + 2 H_2O
\end{equation}

\begin{itemize}
\tightlist
\item
  Fe\textsuperscript{3+}/Fe\textsuperscript{2+}
\end{itemize}

\begin{equation}
Fe^{3+} + 1 e^- \rightleftharpoons Fe^{2+} \\
\end{equation}

\begin{itemize}
\tightlist
\item
  SO\textsubscript{4}\textsuperscript{2-}/H\textsubscript{2}S
\end{itemize}

\begin{equation}
SO_4^{2-} + 8 e^- + 10 H^+  \rightleftharpoons H_2S + 4 H_2O
\end{equation}

\begin{itemize}
\tightlist
\item
  CO\textsubscript{2}/CH\textsubscript{4}
\end{itemize}

\begin{equation}
CO_2 + 8 e^- + 8 H^+ \rightleftharpoons CH_4 + 2 H_2O
\end{equation}

\begin{itemize}
\tightlist
\item
  a redox reaction \textbf{always} involves two half reactions where the
  electrons are transfered from the lowest redox potential couple to the
  highest redox potential couple. For example, the combustion of methane
  in oxygen involves the two O\textsubscript{2}/H\textsubscript{2}O and
  CO\textsubscript{2}/CH\textsubscript{4} couples. The half-reaction of
  the highest redox potential couple goes on top, and that of the lowest
  potential at the bottom, but this time written as red/ox like in the
  equation below:
\end{itemize}

\begin{align}
2\,O_2 + 8\,H^+ + 8\,e^- & \rightleftharpoons & 4\,H_2O\\
CH_4 + 2\,H_2O & \rightleftharpoons & CO_2 + 8\,H^+ + 8\,e^-\\
& &\\
\hline\\
CH_4 + 2\,O_2 & \to & CO_2 + 2\,H_2O
\end{align}

The electrons cancel out, and in this case the protons as well. So the
combustion of methane, which is very simple at first glance, really
involves a transfer of electrons, which the half-reactions reveal.

\hypertarget{reduction}{\subsection{Reduction}\label{reduction}}

\begin{itemize}
\tightlist
\item
  Reduction is the \emph{\textbf{gain}} of electrons by a molecule,
  atom, or ion.
\end{itemize}

\emph{\protect\hyperlink{top}{back to top}}

\section{S}\label{s}

\subsection{Sulfate}\label{sulfate}

\begin{itemize}
\tightlist
\item
  Sulfate is the inorganic sulfur anion stable in oxidized water
\item
  Formula: \(SO_4^{2-}\)
\item
  Sulfate 3D shape:
  \includegraphics[width=0.25000\textwidth]{pictures/Sulfate-3D-balls.png}
\item
  Lewis dot structure of nitrate:
  \includegraphics{pictures/sulfate_lewis_structure.png}
\item
  Number of electron S has for itself following electronegativity rule:
  zero
\item
  \(SO_4^{2-}\) can only be an \emph{\textbf{electron acceptor}}
\item
  \(SO_4^{2-}\) is the conjugate base of Hydrogen sulfate \(HSO_4^{-}\).
  The figure below shows that for pH normally measured in surface water
  and streams (4.5-8), \(SO_4^{2-}\) is the truly preponderant form. We
  therefore generally omit to mention \(HSO_4^{-}\) as a chemical form
  that plays any significant role.
\end{itemize}

\includegraphics{Textbook_BAE204_files/figure-latex/SO4-1.pdf}

\begin{itemize}
\item
  During their anabolism, primary producers uptake sulfate, but the
  sulfur atoms can be incorporated into amino-acids only after sulfur
  has been reduced, or gained 8 electrons to be in a thiol (\(-SH\))
  form.
\item
  Production:
\item
  Consumption:
\item
  Ecological significance:
\item
  Health effects:
\end{itemize}

\emph{\protect\hyperlink{top}{back to top}}

\section{T}\label{t}

\hypertarget{trophic-names}{\subsection{Trophic
names}\label{trophic-names}}

\begin{itemize}
\tightlist
\item
  Depending on the source of carbon, the source of electrons, and the
  source of energy, organisms have been qualified and called
  differently.
\end{itemize}

\begin{center}\rule{0.5\linewidth}{\linethickness}\end{center}

\begin{itemize}
\tightlist
\item
  Depending on the source of \emph{\textbf{carbon}}
\end{itemize}

\textbf{Autotrophs}

\begin{itemize}
\tightlist
\item
  \textbf{Auto}\emph{trophs} obtain their carbon needs, on their own,
  hence \emph{Auto}.
\item
  In other words, they obtain their carbon from CO\textsubscript{2} and
  carbonates
\item
  Just about all green plants and algae are autotrophs
\end{itemize}

\textbf{Heterotrophs}

\begin{itemize}
\tightlist
\item
  \textbf{Hetero}\emph{trophs} obtain their carbon needs, from
  \emph{others}, hence \emph{Hetero}.
\item
  In other words, they obtain their carbon from organic matter, some of
  which might also serve as a source of electrons
\item
  All animals are heterotrophs, and most bacteria are too.
\end{itemize}

\begin{center}\rule{0.5\linewidth}{\linethickness}\end{center}

\begin{itemize}
\tightlist
\item
  Depending on the source of \emph{\textbf{electrons}}
\end{itemize}

\textbf{Lithotrophs}

\begin{itemize}
\tightlist
\item
  \textbf{Litho}\emph{trophs} obtain their reducing power or their high
  energy electrons from inorganic molecules, or \emph{mineral matter}.
  Literally, \emph{rock eaters}.
\item
  In other words, they obtain their electrons from molecules or atoms
  like NH\textsubscript{4}\textsuperscript{+} or Fe\textsuperscript{2+}.
\item
  ALL Lithotrophs are unicellular microorganisms
\item
  In the examples above, nitrifiers and iron oxidizing bacteria are
  lithotrophs
\end{itemize}

\textbf{Organotrophs}

\begin{itemize}
\tightlist
\item
  \textbf{Organo}\emph{trophs} obtain their reducing power or their high
  energy electrons from \emph{organic} molecules, hence the
  \emph{organo}\\
\item
  In other words, they obtain their electrons from organic matter
\item
  All pluricellular organisms are organotrophs 
\end{itemize}

\begin{center}\rule{0.5\linewidth}{\linethickness}\end{center}

\begin{itemize}
\tightlist
\item
  Depending on the source of \emph{\textbf{energy}}
\end{itemize}

\textbf{Phototrophs}

\begin{itemize}
\tightlist
\item
  \textbf{Photo}\emph{trophs} obtain their energy primarily through
  sunlight.
\item
  In reality, at night, phototrophs become chemotrophs as their cellular
  respiration is based upon oxidizing organic matter. But they are
  autosufficient. They capture solar energy and store it in a chemical
  form thanks to photosynthesis, and used that stored energy later on.
\item
  Just about all green plants are phototrophs.
\end{itemize}

\textbf{Chemotrophs}

\begin{itemize}
\tightlist
\item
  \textbf{Chemo}\emph{trophs} obtain their energy as energy previously
  stored in a chemical form.
\item
  Just about all pluricellular organisms are chemotrophs, even the
  phototrophs, which turn into chemotrophs at night 
\end{itemize}

\bibliography{book.bib,packages.bib}


\end{document}
